\section{Основные понятия}
\selectlanguage{russian}
Для успешного выполнения любых целей по защите информации необходимо участие в процессе защиты нескольких субъектов, которые по определённым правилам будут выполнять технические или организационные действия, криптографические операции, взаимодействовать друг с другом, например, передавая сообщения или проверяя личности друг друга.

Формализация подобных действий делается через описание протокола. \emph{Протокол} -- описание распределённого алгоритма, в процессе выполнения которого два или более участников последовательно выполняют определённые действия и обмениваются сообщениями\footnote{Здесь и далее в этом разделе определения даны на основе~\cite{Cheremushkin:2009}.}.

Под участником\index{участник!протокола} (субъектом\index{субъект!протокола}, стороной\index{сторона!протокола}) протокола понимают не только людей, но и приложения, группы людей или целые организации. Формально участниками считают только тех, кто выполняет активную роль в рамках протокола. Хотя при создании и описании протоколов забывать про пассивные стороны тоже не стоит. Например, пассивный криптоаналитик\index{криптоаналитик!пассивный} формально не является участником протоколов, но многие протоколы разрабатываются с учётом защиты от таких <<неучастников>>.

Протокол состоит из \emph{циклов}\index{цикл!протокола} (\langen{round}) или \emph{проходов}\index{проход!протокола} (\langen{pass}). Цикл -- временной интервал активности только одного участника. За исключением самого первого цикла протокола, обычно начинается приёмом сообщения, а заканчивается -- отправкой.

Цикл (или проход) состоит из \emph{шагов} (действий, \langen{step, action}) -- конкретных законченных действий, выполняемых участником протокола. Например:
\begin{itemize}
	\item генерация нового (случайного) значения;
	\item вычисление значений функции;
	\item проверка сертификатов, ключей, подписей, и др.;
	\item приём и отправка сообщений.
\end{itemize}

Прошедшая в прошлом или даже просто теоретически описанная реализация протокола для конкретных участников называется \emph{сеансом}\index{сеанс!протокола}. Каждый участник в рамках сеанса выполняет одну или несколько \emph{ролей}. В другом сеансе протокола участники могут поменяться ролями и выполнять уже совсем другие функции.

Можно сказать, что протокол прескрептивно описывает правила поведения каждой роли в протоколе. А сеанс это дескриптивное описание (возможно теоретически) состоявшейся в прошлом реализации протокола.

Пример описания протокола.
\begin{enumerate}
	\item Участник с ролью <<Отправитель>> должен отправить участнику с ролью <<Получатель>> сообщение.
	\item Участник с ролью <<Получатель>> должен принять от участника с ролью <<Отправитель>> сообщение.
\end{enumerate}

Пример описания сеанса протокола.
\begin{enumerate}
	\item 1-го апреля в 13:00 Алиса отправила Бобу сообщение.
	\item 1-го апреля в 13:05 Боб принял от Алисы сообщение.
\end{enumerate}

\emph{Защищённым протоколом}\index{протокол!защищённый} или \emph{протоколом обеспечения безопасности}\index{протокол!обеспечения безопасности} будет называть протокол, обеспечивающий выполнение хотя бы одной защитной функции~\cite{ISO:7498-2:1989}:
\begin{itemize}
	\item аутентификация сторон и источника данных,
	\item разграничение доступа,
	\item конфиденциальность,
	\item целостность,
	\item невозможность отказа от факта отправки или получения.
\end{itemize}

Если защищённый протокол предназначен для выполнения функций безопасности криптографической системы, или если в процессе его исполнения используются криптографические алгоритмы, то такой протокол будем называть \emph{криптографическим}\index{протокол!криптографический}.
