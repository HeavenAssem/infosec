\subsection{Протокол Диффи~---~Хеллмана}\index{протокол!Диффи~---~Хеллмана}\label{section-protocols-diffie-hellman}
\selectlanguage{russian}

Алгоритм с открытым ключом впервые был предложен Диффи и Хеллманом в работе 1976 года <<Новые направления в криптографии>> (\langen{Bailey Whitfield Diffie, Martin Edward Hellman, ``New directions in cryptography''},~\cite{Diffie:Hellman:1976}).

Рассмотрим протокол Диффи~---~Хеллмана обмена информацией двух сторон $A$ и $B$. Задача состоит в том, чтобы создать общий сеансовый ключ.

Пусть $p$ -- большое простое число\index{число!простое}, $g$ -- примитивный элемент группы $\Z_p^*$, ~ $y = g^x \mod p$, причём $p,y,g$ известны заранее. Функцию $y=g^{x} \mod p$ считаем однонаправленной, то есть вычисление функции при известном значении аргумента является лёгкой задачей, а её обращение (нахождение аргумента) при известном значении функции -- трудной.

Протокол обмена состоит из следующих действий.
\begin{enumerate}
    \item[(1)] Алиса выбирает случайное $2 \leq a \leq p - 1$
    \item[{}] $Alice \to \left\{ A = g ^ x \bmod p \right\} \to Bob$
    \item[(2)] Боб выбирает случайное $2 \leq b \leq p-1$
    \item[{}] Боб вычисляет сеансовый ключ $K = A ^ b \bmod p$
    \item[{}] $Bob \to \left\{ B = g ^ b \bmod p \right\} \to Alice$
    \item[{}] Алиса вычисляет $K = B ^ a \bmod p$
\end{enumerate}

Таким способом создан общий секретный сеансовый ключ $K$. За счёт случайного выбора значений $a$ и $b$ в новом сеансе будет получен новой сеансовый ключ.

Данный протокол обеспечивает только генерацию новых сеансовых ключей (цель G10). В отсутствие третей доверенной стороны он не обеспечивает ни аутентификацию сторон (цель G1), из-за отсутствия проходов с подтверждением владения ключом отсутствует аутентификация ключа (цель G8). Зато, так как протокол не использует длительные <<мастер>>-ключи, можно говорить о том, что он обладает свойством совершенной прямой секретности (цель G9).

Данный протокол можно использовать только с такими каналами связи, в которые не может вмешаться активный криптоаналитик. В противном случае протокол становится уязвим к простой <<атаке посередине>>.

\begin{enumerate}
    \item[(1)] Алиса выбирает случайное $2 \leq a \leq p - 1$
    \item[{}] $Alice \to \left\{ A = g ^ x \bmod p \right\} \to Mellory~(Bob)$
    \item[(2)] Меллори выбирает случайное $2 \leq m \leq p-1$
    \item[{}] Меллори вычисляет сеансовый ключ для канала с Алисой $K_{AM} = A ^ m \bmod p = g ^ {am} \bmod p$
    \item[{}] $Mellory~(Bob) \to \left\{ M = g ^ m \bmod p \right\} \to Alice$
    \item[{}] Алиса вычисляет сеансовый ключ для канала с Меллори (думая, что Меллори это Боб) $K_{AM} = M ^ a \bmod p = g ^ { am } \bmod p$
    \item[{}] $Mellory~(Alice) \to \left\{ M = g ^ m \bmod p \right\} \to Bob$
    \item[(3)] Боб выбирает случайное $2 \leq b \leq p-1$
    \item[{}] Боб вычисляет сеансовый ключ для канала с Меллори (думая, что Меллори это Алиса) $K_{BM} = M ^ b \bmod p = g ^ { bm } \bmod p$
    \item[{}] $Bob \to \left\{ B = g ^ b \bmod p \right\} \to Mellory~(Alice)$
    \item[{}] Меллори вычисляет сеансовый ключ для канала с Бобом $K_{BM} = B ^ m \bmod p = g ^ { bm } \bmod p$
\end{enumerate}

В результате Алиса и Боб получили новые сеансовые ключи, но <<защищённый>> канал связи установили не с друг с другом, а со злоумышленником, который теперь имеет возможность ретранслировать или изменять все передаваемые сообщения между Алисой и Бобом.
