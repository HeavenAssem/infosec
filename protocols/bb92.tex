\subsection{Протокол B92 (BB92)}\index{протокол!B92|(}\index{протокол!BB92|(}
\selectlanguage{russian}

В 1992 году один из авторов протокола BB84 Чарльз Беннетт выдвинул идею (\cite{Bennett:1992}), что участникам не обязательно использовать четыре разных варианта поляризации, а достаточно двух, но неортогональных. Например, $0^{\circ}$ (<<$\to$>>) и $45^{\circ}$ (<<$\nearrow$>>). Протокол получил названия B92 и BB92.

Для каждого бита выполняется следующее.

\begin{protocol}
    \item[(1)] Алиса поляризует фотон в зависимости от бита $b_i$ и передаёт его по квантовому каналу связи Бобу:
        \begin{itemize}
            \item для $b_i=0$ поляризует под $0^{\circ}$ (<<$\to$>>);
            \item для $b_i=1$ поляризует под $45^{\circ}$ (<<$\nearrow$>>).
        \end{itemize}
    \item[(2)] Боб случайным образом выбирает один из двух базисов: $90^{\circ}$ (<<$\uparrow$>>) или $135^{\circ}$ (<<$\nwarrow$>>), и пробует детектировать фотон. Если получилось, то он делает вывод о выбранной Алисой поляризации фотона и бите $b_i$:
        \begin{itemize}
            \item если детектировал на $135^{\circ}$ (<<$\nwarrow$>>), значит Алиса выбрала поляризацию $0^{\circ}$ (<<$\to$>>) и $b_i=0$;
            \item если детектировал на $90^{\circ}$ (<<$\uparrow$>>), значит Алиса выбрала поляризацию $45^{\circ}$ (<<$\nearrow$>>) и $b_i=1$.
        \end{itemize}
    \item[{}] Боб по открытому классическому каналу связи сообщает Алисе, получилось детектировать фотон или нет. Если да, то бит принимается участниками за переданный.
\end{protocol}

\begin{table}
    \centering
    \begin{tabular}{|l|c|c|c|c|c|c|c|c|}
        \hline
        \parbox[c][1cm][c]{2,8cm}{биты Алисы} & 0 & 0 & 0 & 0 & 1 & 1 & 1 & 1 \\
        \hline
        \parbox[c][1cm][c]{2,8cm}{поляризация \\ фотона} & $\to$ & $\to$ & $\to$ & $\to$ & $\nearrow$ & $\nearrow$ & $\nearrow$ & $\nearrow$ \\
        \hline
        \parbox[c][1cm][c]{2,8cm}{поляризация \\ детектора Боба} & $\nwarrow$ & $\uparrow$ & $\nwarrow$ & $\uparrow$ & $\nwarrow$ & $\uparrow$ & $\nwarrow$ & $\uparrow$ \\
        \hline
        \parbox[c][1cm][c]{2,8cm}{вероятность детектирования} & $\frac{1}{2}$ & 0 & $\frac{1}{2}$ & 0 & 0 & $\frac{1}{2}$ & 0 & $\frac{1}{2}$ \\
        \hline
        \parbox[c][1cm][c]{2,8cm}{удалось или нет детектировать} & да & нет & нет & нет & нет & да & нет &  нет \\
        \hline
        \parbox[c][1cm][c]{2,8cm}{принятые Бобом биты} & 0 & - & - & - & - & 1 & - & - \\
        \hline
    \end{tabular}
    \caption{Пример сеансов протокола B92 / BB92. По итогам передачи 8 фотонов от Алисы Боб сумел детектировать 2 фотона, что позволило передать от Алисы к Бобу 2 бита информации}
    \label{tab:b92}
\end{table}

Если Боб выбрал поляризацию, ортогональную оригинальной, то он со 100\% вероятностью не зарегистрирует фотон. Если же поляризация неортогональна оригинальной, то с вероятностью 50\% Боб сумеет зарегистрировать фотон на детекторе. Таким образом, если Боб зарегистрировал фотон, то он будет точно знать, какой бит передавала Алиса. Если же не зарегистрировал, то это трактуется, как стирание.

Пример сеансов протоколов передачи приведён в таблице~\ref{tab:b92}.

В среднем для передачи одного бита информации Алисе и Бобу потребуется провести 4 итерации протокола. Это в два раза больше, чем в протоколе BB84\index{протокол!BB84}.

\index{протокол!B92|)}\index{протокол!BB92|)}
