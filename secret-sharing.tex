\chapter{Разделение секрета}\index{схема!разделения секрета|(}
\selectlanguage{russian}

\section{Пороговые схемы}

Идея \emph{пороговой} $(K, N)$-схемы\index{разделение секрета!пороговое} разделения общего секрета среди $N$ пользователей состоит в следующем. Доверенная сторона хочет распределить некий секрет $K_0$ между $N$ пользователями таким образом, что:
\begin{itemize}
    \item любые $m_1: K \leq m_1 \leq N$, легальных пользователей могут получить секрет (или доступ к секрету), если предъявят свои секретные ключи;
    \item любые $m_2: m_2 < K$, легальных пользователей не могут получить секрет и не могут определить (вычислить) этот секрет, даже решив трудную в вычислительном смысле задачу.
\end{itemize}

Далее рассмотрены три случая: $(K, N)$-схема Блэкли, $(K, N)$-схема Шамира и простая $(N,N)$-схема.

\input{secret-sharing-blackleys}

\input{secret-sharing-shamirs}

\input{secret-sharing-xor}

\section{Распределение секрета по коалициям}

\subsection{Схема для нескольких коалиций}

Предположим, что имеется $N$ легальных пользователей
    \[ \{ U_1, U_2, \dots, U_N \}, \]
которым нужно сообщить (открыть, предоставить в доступ) общий секрет $K$.

Секрет может быть доступен только определённым коалициям\index{распределение секрета!по коалициям}, например:
\[ \begin{array}{l}
    C_1 = \{ U_1, U_2 \}, \\
    C_2 = \{ U_1, U_3, U_4 \}, \\
    C_3 = \{ U_2, U_3 \}, \\
    \dots
\end{array} \]
При этом ни одна из коалиций $C_i, ~ i = 1, 2, \dots$ не должна быть подмножеством другой коалиции.


\example
Имеется 4 участника:
    \[ \{ U_1, U_2, U_3, U_4 \}, \]
которые образуют 3 коалиции:
\[ \begin{array}{l}
    C_1 = \{ U_1, U_2 \}, \\
    C_2 = \{ U_1, U_3 \}, \\
    C_3 = \{ U_2, U_3, U_4 \}. \\
\end{array} \]
Распределение частичных секретов между ними представлено в виде таблицы~\ref{tab:secret-share-coalition-1}, в которой введены следующие обозначения: $a_1, b_1, c_2, c_3$ -- случайные числа из кольца $\Z_M$. В строках таблицы содержатся частичные секреты каждого из пользователей, в столбцах таблицы показаны частичные секреты, соответствующие каждой из коалиций.

\begin{table}[!ht]
    \centering
    \caption{Распределение секрета по определённым коалициям\label{tab:secret-share-coalition-1}}
    \begin{tabular}{|c||c|c|c|}
        \hline
              & $C_1 = \{ U_1, U_2 \}$ & $C_2 = \{U_1, U_3 \}$ & $C_3 = \{ U_2, U_3, U_4 \}$ \\
        \hline \hline
        $U_1$ & $a_1$     & $b_1$     & -- \\
        $U_2$ & $K - a_1$ & --        & $c_2$ \\
        $U_3$ & --        & $K - b_1$ & $c_3$  \\
        $U_4$ & --        & --        & $K - c_2 - c_3$ \\
        \hline
    \end{tabular}
\end{table}

Как видно из приведённых данных, суммирование по модулю $M$ чисел, записанных в каждом из столбцов таблицы, открывает секрет $K$.
\exampleend


\example

%\section{Схема разделения секрета на монотонных булевых функциях}
%\example
В системе распределения секрета доверенный
%с использованием монотонных булевых функций
центр использует кольцо $\Z_m$ целых чисел по модулю $m$. Требуется разделить секрет $K$ между $5$ пользователями:
    \[ \{ U_1, U_2, U_3, U_4, U_5 \} \]
так, чтобы восстановить секрет могли только коалиции:
\[ \begin{array}{lll}
    C_1 = \{ U_1, U_2 \},      & & C_2 = \{ U_1, U_3 \}, \\
    C_3 = \{ U_2, U_3, U_4 \}, & & C_4 = \{ U_2, U_3, U_5 \}, \\
    C_5 = \{ U_3, U_4, U_5 \}, & & C_6 = \{ U_1, U_2, U_3 \}. \\
\end{array} \]

Заданное множество коалиций с доступом не является минимальным, так как одни коалиции входят в другие:
    \[ C_1 \subset C_6, ~ C_2 \subset C_6. \]
Исключая $C_6$, получим минимальное множество коалиций с доступом к секрету: ни одна из оставшихся коалиций не входит в другую $C_i \nsubseteq C_j$ для $i \neq j$. Пользователям выдаются тени по минимальному множеству коалиций с доступом. В строках таблицы~\ref{tab:secret-share-coalition-2} содержатся частичные секреты каждого из пользователей, в столбцах таблицы показаны частичные секреты, соответствующие каждой из коалиций.

\begin{table}[!ht]
    \centering
    \caption{Распределение секрета по определённым коалициям\label{tab:secret-share-coalition-2}}
    \begin{tabular}{|c||c|c|c|c|c|}
        \hline
              & $C_1$     & $C_2$     & $C_3$           & $C_4$           & $C_5$  \\
        \hline \hline
        $U_1$ & $a_1$     & $b_1$     & --              & --              & -- \\
        $U_2$ & $K - a_1$ & --        & $c_2$           & $d_2$           & --\\
        $U_3$ & --        & $K - b_1$ & $c_3$           & $d_3$           & $e_3$ \\
        $U_4$ & --        & --        & $K - c_2 - c_3$ & --              & $e_4$ \\
        $U_5$ & --        & --        & --              & $K - d_2 - d_3$ & $K - e_3 - e_4$ \\
        \hline
    \end{tabular}
\end{table}

Тени выбираются случайно из кольца $\mathbb{\Z}_m$. В результате у пользователей будут тени. 
\exampleend

\subsection{Схема Брикелла для нескольких коалиций}\index{схема разделения секрета!Бриккела|(}\index{схема!Бриккела|(}
\selectlanguage{russian}

Рассмотрим схему Брикелла (\langen{Ernest Francis Brickell},~\cite{Brickell:1990}) разделения секрета по коалициям.

По-прежнему
    \[ \{ U_1, U_2, \dots, U_N \} \]
-- легальные пользователи. Пусть $\Z_p$ -- кольцо целых чисел по модулю $p$. Рассмотрим векторы
    \[ \mathcal{U} = \left\{ (u_1, u_2, \dots, u_d) \right\}, ~~ u_i \in \Z_p \]
длины $d$. Каждому пользователю $U_i, ~ i = 1, \dots, N$ ставится в соответствие вектор
    \[ \varphi(U_i) \in \mathcal{U}, ~~ i = 1, \dots, N. \]

Тогда каждой из коалиций, например
    \[ C_1 = \{ U_1, U_2, U_3 \}, \]
соответствует набор векторов
    \[ \varphi(U_1), \varphi(U_2), \varphi(U_3). \]
Эти векторы должны быть выбраны так, чтобы их линейная оболочка \emph{содержала} вектор
    \[ (1, 0, 0, \dots, 0) \]
длины $d$. Линейная оболочка любого набора векторов, не образующих коалицию, \emph{не должна} содержать вектор $(1, 0, 0, \dots, 0)$ длины $d$.

Пусть $K_0 \in \Z_p$ -- общий секрет. Распределение секрета производится следующим образом. Сначала вычисляется вектор $(K_0, K_1, \dots, K_{d-1})$, где первая координата -- это общий секрет, а остальные координаты выбираются из $\Z_p$ случайно. Затем вычисляются скалярные произведения:
\[\begin{array}{l}
	\left( \left( K_0, K_1, \dots, K_{d-1} \right), ~ \varphi(U_1) \right) ~=~ a_1, \\
	\left( \left( K_0, K_1, \dots, K_{d-1} \right), ~ \varphi(U_2) \right) ~=~ a_2, \\
	\dots \\
	\left( \left( K_0, K_1, \dots, K_{d-1} \right), ~ \varphi(U_N) \right) ~=~ a_N. \\
\end{array}\]

Пользователям $U_i, ~ i = 1, 2, \dots, N$ выдаются их частичные секреты:
    \[ U_i \colon \left\{ \varphi(U_i), a_i \right\}. \]

Пусть коалиция $C$ -- допустимая, например:
    \[ C = C_1 = \{ U_1, U_2, U_3 \}. \]

Тогда члены коалиции совместно находят такие коэффициенты $\lambda_1, \lambda_2, \lambda_3$, что
    \[ \lambda_1\varphi(U_1)+\lambda_2\varphi(U_2)+\lambda_3\varphi(U_3) ~=~ (1,0, \dots, 0). \]

После этого вычисляется выражение
\[\begin{array}{l}
    \lambda_1 a_1 + \lambda_2 a_2 + \lambda_3 a_3 = \\
    = \left( \left( K_0, K_1, \dots, K_{d-1} \right), ~ \lambda_1 \varphi(U_1) + \lambda_2 \varphi(U_2) + \lambda_3 \varphi(U_3) \right) = \\
    = \left( \left( K_0, K_1, \dots, K_{d-1} \right), ~ \left( 1, 0, \dots, 0 \right) \right) =  K_0, \\
\end{array}\]
которое и является общим секретом.

%\section{Схема разделения секрета в векторном пространстве Бриккела}
%
%В схеме Бриккела для $n$ пользователей $\{ U_1, U_2, \ldots, U_n \}$ Центр выбирает $k$-мерные векторы $\varphi(U_i)$ над полем $\mathbb{\Z}_p$ так, чтобы их линейная нетривиальная комбинация над полем $\mathbb{\Z}_p$ могла равняться единичному вектору
%    \[ (1,0,0, \ldots, 0) = \sum\limits_{i=1}^{n} c_i \ \varphi(U_i), ~ c_i \in  \mathbb{\Z}_p. \]
%Центр пользователю $i$ присваивает \emph{открытый, публично доступный} вектор $\varphi(U_i)$.
%
%Для разделения секрета $K$ Центр выбирает случайные числа $a_2, a_3, \ldots, a_n$, составляет вектор $\bar{a} = (K, a_2, a_3, \ldots, a_n)$ и выдаёт каждому пользователю \emph{секретную} тень $s_i = \bar{a} \cdot \varphi(U_i)$.
%
%Восстановление секрета производится
%    \[ K = \bar{a} \cdot (1,0,\ldots,0) = \bar{a} \cdot \sum\limits_{i=1}^{n} c_i \varphi(U_i) = \sum\limits_{i=1}^{n} c_i s_i, \]
%так как из открытых векторов $\varphi(U_i)$ пользователи могут найти $c_i$.
%
%Восстановить секрет могут только те коалиции пользователей, для которых нетривиальная комбинация векторов $\varphi(U_i)$ даёт единичный вектор.

\example
Для сети из $n = 4$ участников
    \[ \{ U_1, U_2, U_3, U_4 \} \]
выбраны следующие векторы длины $k = 3$ над полем $\Z_{23}$:
\[ \begin{array}{l}
    \varphi(U_1) = (0,2,0), \\
    \varphi(U_2) = (2,0,7), \\
    \varphi(U_3) = (0,5,7), \\
    \varphi(U_4) = (0,2,9). \\
\end{array} \]
Найдём все коалиции, которые могут раскрыть секрет.

Запишем
    \[ (1,0,0) = c_1 (0,2,0) + c_2 (2,0,7) + c_3 (0,5,7) + c_4 (0,2,9). \]
Ясно, что $c_2 \neq 0$ и коалициями пользователей, которые дают единичный вектор и, следовательно, могут восстановить секрет, являются:
\[ \begin{array}{l}
    C_1 = \{ U_1, U_2, U_3 \}, \\
    C_2 = \{ U_1, U_2, U_4 \}, \\
    C_3 = \{ U_2, U_3, U_4 \}. \\
\end{array} \]

Пусть доверенный центр для секрета $K = 4$ выбрал вектор $\bar{a} = (4, 2, 9)$. Тогда участники получают тени:
    \[ s_1 = (4,2,9) \cdot (0,2,0) = 4 \mod 23, \]
    \[ s_2 = (4,2,9) \cdot (2,0,7) = 2 \mod 23, \]
    \[ s_3 = (4,2,9) \cdot (0,5,7) = 4 \mod 23, \]
    \[ s_4 = (4,2,9) \cdot (0,2,9) = 16 \mod 23. \]

Возьмём коалицию $C_1 = \{ U_1, U_2, U_3 \}$ и вычислим коэффициенты $c_i$:
    \[ (1,0,0) = c_1 (0,2,0) + c_2 (2,0,7) + c_3 (0,5,7), \]
\[ \begin{array}{l}
    c_1 = 7 \mod 23, \\
    c_2 = 12 \mod 23, \\
    c_3 = 11 \mod 23. \\
\end{array} \]

Найдём секрет:
    \[ K = 7 \cdot 4 + 12 \cdot 2 + 11 \cdot 4 = 4 \mod 23.\]
\exampleend

\index{схема разделения секрета!Бриккела|)}\index{схема!Бриккела|)}


\index{схема!разделения секрета|)}