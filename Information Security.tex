\documentclass[10pt,a4paper,openany]{book}
% Векторные русских шрифтов в PDF
% Необходимо установить также пакеты cm-super и cm-unicode
\usepackage{cmap}
\usepackage[X2, T2A]{fontenc}
\usepackage[utf8]{inputenc}
\usepackage[english,french,german,italian,russian]{babel}
\usepackage{indentfirst}                % Красная строка в первом абзаце
% Подключение amsmath также даёт поддержку автоматических \dots
% см. http://tex.stackexchange.com/questions/77737/dots-versus-ldots-is-there-a-difference
% см. http://tex.stackexchange.com/questions/117730/what-is-the-difference-between-ldots-and-cdots
\usepackage{amsmath} % разрешить \texttt и аналогичные в формулах
\usepackage{amssymb} % дополнительные математические символы
\usepackage{graphicx} % поддержка изображений

%\usepackage{amsfonts, eucal, bm, color, }

\usepackage{algorithm, algorithmic}     % 'algorithm' environments
\floatname{algorithm}{Алгоритм}

\usepackage{array}                      % Новые команды для работы с таблицами, см. также ниже доопределение стилей ячеек L, C, R
\usepackage{arydshln}                   % dash lines in tables
\usepackage{caption}                    % titles for figures
\usepackage{csquotes}
\usepackage{enumerate}
\usepackage{enumitem}                   % кастомизация itemize/enumerate, напр. отказ от indent
\usepackage{fancybox}                   % страница в рамке
\usepackage{float}			% sub figures
\usepackage[totoc=true]{idxlayout}      % балансировка индексов на последней странице, индекс в ToC
\usepackage{lscape}                     % поддержка поворота страниц на 90 градусов для широких таблиц
\usepackage{multicol}                   % поддержка колонок
\setlength{\columnsep}{0.1cm}
\usepackage{multirow}                   % multirow cells in tables
%\usepackage{subfig}			% sub figures
\usepackage{subcaption}         % sub figures, incompatible with package{subfig}
\usepackage{tikz}                       % векторная графика внутри TeX
\usepackage{tablefootnote}		% footnote в таблицах
\usepackage{wrapfig}			% sub figures
\usepackage{textcomp}                   % \No support
\usepackage{cjhebrew}
\usepackage{import}

\usepackage[left=1.84cm, right=1.5cm, paperwidth=14cm, top=1.8cm, bottom=2cm, height=19.8cm, paperheight=20cm]{geometry}

% Необходимо установить biblatex-gost
% и переключить сборку библиографии на biber
\usepackage[parentracker=true,
  backend=biber,
  hyperref=auto,
  language=auto,
  autolang=other,
  citestyle=gost-numeric,
  defernumbers=true,
  bibstyle=gost-numeric,
  sortlocale=ru_RU
]{biblatex}								% библиография по ГОСТу

\newcommand{\indexchapter}[1]{%
\chapter*{#1}%
\addcontentsline{toc}{chapter}{#1}%
}
\usepackage[xindy]{imakeidx}
\makeindex[program=texindy, options=-M mystyle.xdy -L russian -C utf8]
\indexsetup{level=\indexchapter,toclevel=chapter}

\let\oldindex\index

\makeatletter
\renewcommand{\index}[2][\imki@jobname]{%
	\oldindex[#1]{\detokenize{#2}}%
}
\makeatother

% поддержка гиперссылок; гиперссылки в pdf, должен быть последним загруженным пакетом
\ifx\pdfoutput\undefined
    \usepackage[unicode,dvips]{hyperref}
\else
    \usepackage[pdftex,colorlinks,unicode,bookmarks]{hyperref}
\fi

%\paperwidth=16.8cm \oddsidemargin=0cm \evensidemargin=0cm \hoffset=-0.33cm \textwidth=13.2cm
%\paperheight=24cm \voffset=-0.4cm \topmargin=0cm \headsep=0cm \headheight=0cm \textheight=19.8cm \footskip=0.9cm

% параметры PDF файла
\hypersetup{
    pdftitle={Защита информации},
    pdfauthor={Э. М. Габидулин, А. С. Кшевецкий, А. И. Колыбельников, С. М. Владимиров},
    pdfsubject=учебное пособие,
    pdfkeywords={защита информации, криптография, МФТИ}
}

% добавить точку после номера секции, раздела и~т.\,д.
\makeatletter
\def\@seccntformat#1{\csname the#1\endcsname.\quad}
\def\numberline#1{\hb@xt@\@tempdima{#1\if&#1&\else.\fi\hfil}}
\makeatother

% перенос слов с тире
%\lccode`\-=`\-
%\defaulthyphenchar=127

% изменить подписи к рисункам, таблицам и~т.\,д.
\captionsetup{labelsep=endash}          % Нумерационный заголовок и текст разделяются тире
\captionsetup{textformat=simple}        % Текст подписи будет напечатан как есть
%\captionsetup[table]{position=above}    % вертикальные отступы подписи таблицы для случая, когда подпись вверху
%\captionsetup[figure]{position=below}   % вертикальные отступы подписи рисунка для случая, когда подпись внизу

%% стиль главы и секции вверху страницы
%\pagestyle{fancy}
%%\renewcommand{\chaptermark}[1]{\markboth{#1}{}}
%\renewcommand{\sectionmark}[1]{\markright{#1}{}}
%
%%\fancyhf{}
%%\fancyfoot[СE,CO]{\thepage}
%%\fancyhead[LE]{\textsc{\nouppercase{\leftmark}}}
%\fancyhead[RO]{\textsc{\nouppercase{\rightmark}}}
%
%\fancypagestyle{plain}{ %
%\fancyhf{}                              % remove everything
%\renewcommand{\headrulewidth}{0pt}      % remove lines as well
%\renewcommand{\footrulewidth}{0pt}}

% запретить выходить за границы страницы
\sloppy

\newtheorem{theorem}{Теорема}[section]
\newtheorem{lemma}[theorem]{Лемма}
\newtheorem{definition}[theorem]{Определение}
\newtheorem{property}[theorem]{Утверждение}
\newtheorem{corollary}[theorem]{Следствие}
%\newtheorem{algorithm}[theorem]{Алгоритм}
\newtheorem{remark}[theorem]{Замечание}
\newcommand{\proof}{\noindent\textsc{Доказательство.\ }}

%\newtheorem{example}{\textsc{\textbf{Пример}}}
\newcommand{\example}{\textsc{\textbf{Пример.}} }
\newcommand{\exampleend}

\DeclareMathOperator{\ord}{ord}
\newcommand{\set}[1]{\mathbb{#1}}
\newcommand{\group}[1]{\mathbb{#1}}
\newcommand{\E}{\group{E}}
\newcommand{\F}{\group{F}}
\newcommand{\GF}[1]{\group{GF}(#1)}
\newcommand{\Gr}{\group{G}}
\newcommand{\Mod}{\operatorname{mod}}
\newcommand{\R}{\group{R}}
\newcommand{\Z}{\group{Z}}
\newcommand{\MAC}{\textrm{MAC}}
\newcommand{\HMAC}{\textrm{HMAC}}
\newcommand{\PK}{\textrm{PK}}
\newcommand{\SK}{\textrm{SK}}

\newcommand{\langde}[1]{нем. \foreignlanguage{german}{\textit{#1}}}
\newcommand{\langfr}[1]{фр. \foreignlanguage{french}{\textit{#1}}}
\newcommand{\langen}[1]{англ. \foreignlanguage{english}{\textit{#1}}}
\newcommand{\langit}[1]{итал. \foreignlanguage{italian}{\textit{#1}}}
\newcommand{\langlat}[1]{лат. \foreignlanguage{italian}{\textit{#1}}}

% Русская типографика
\renewcommand\leq{\leqslant}
\renewcommand\geq{\geqslant}
\renewcommand\emptyset{\varnothing}
\renewcommand\kappa{\varkappa}
\renewcommand\epsilon{\varepsilon}
\renewcommand\phi{\varphi}
\renewcommand*{\No}{\textnumero}

% Для раздела с задачами
\newcommand{\taskinit}{\newcounter{task-section}\setcounter{task-section}{0}\newcounter{task-number}}
\newcommand{\tasksection}{\addtocounter{task-section}{1}\setcounter{task-number}{0}}
\newcommand{\tasknumber}{\textbf{\No\addtocounter{task-number}{1}\arabic{task-section}.\arabic{task-number}.}~~}

%Наконец, существует способ дублировать знаки операций, который мы приведём безо всяких пояснений. Включив
%\newcommand*{\hm}[1]{#1\nobreak\discretionary{}{\hbox{\mathsurround=0pt #1}}{}}
%в преамбулу, можно написать $a\hm+b\hm+c\hm+d$, при этом в формуле a\hm+b\hm+c\hm+d при переносе знак + будет продублирован.

% Дублирование символов бинарных операций ("+", "-", "="), набранных в строчных формулах, при переносе на другую строку:
%%begin{latexonly}
%\renewcommand\ne{\mathchar"3236\mathchar"303D\nobreak
%      \discretionary{}{\usefont
%      {OMS}{cmsy}{m}{n}\char"36\usefont
%      {OT1}{cmr}{m}{n}\char"3D}{}}
%\begingroup
%\catcode`\+\active\gdef+{\mathchar8235\nobreak\discretionary{}%
% {\usefont{OT1}{cmr}{m}{n}\char43}{}}
%\catcode`\-\active\gdef-{\mathchar8704\nobreak\discretionary{}%
% {\usefont{OMS}{cmsy}{m}{n}\char0}{}}
%\catcode`\=\active\gdef={\mathchar12349\nobreak\discretionary{}%
% {\usefont{OT1}{cmr}{m}{n}\char61}{}}
%\endgroup
%\def\cdot{\mathchar8705\nobreak\discretionary{}%
% {\usefont{OMS}{cmsy}{m}{n}\char1}{}}
%\def\times{\mathchar8706\nobreak\discretionary{}%
% {\usefont{OMS}{cmsy}{m}{n}\char2}{}}
%\mathcode`\==32768
%\mathcode`\+=32768
%\mathcode`\-=32768
%%end{latexonly}

% Упрощение создания таблиц с длинным текстом в ячейках
\newcolumntype{L}[1]{>{\raggedright\let\newline\\\arraybackslash\hspace{0pt}}m{#1}}
\newcolumntype{C}[1]{>{\centering\let\newline\\\arraybackslash\hspace{0pt}}m{#1}}
\newcolumntype{R}[1]{>{\raggedleft\let\newline\\\arraybackslash\hspace{0pt}}m{#1}} 

% Рекомендация для latex'а не разрывать inline-формулы
\binoppenalty=9999
\relpenalty=9999


\addbibresource{bibliography.bib}

\title{Криптографические методы \\ защиты информации \\ \bigskip \normalsize{Учебное пособие}}
\author{Владимиров Сергей Михайлович \\ Габидулин Эрнст Мухамедович \\ Колыбельников Александр Иванович \\ Кшевецкий Александр Сергеевич}
\date{\bigskip \bigskip \bigskip \bigskip \bigskip \today \bigskip \\ \small{черновой вариант третьего издания}}

\begin{document}
\selectlanguage{russian}

\maketitle
\setcounter{page}{3}

\newpage
%\thispagestyle{empty}
\setcounter{tocdepth}{2}
\tableofcontents
%\thispagestyle{empty}
\newpage

%\lhead[\leftmark]{}
%\rhead[]{\rightmark}

\chapter*{Предисловие}
\addcontentsline{toc}{chapter}{Предисловие}
\markboth{ПРЕДИСЛОВИЕ}{ПРЕДИСЛОВИЕ}
\selectlanguage{russian}

В настоящем пособии рассмотрены только основные математические методы защиты информации, и среди них главный акцент сделан на криптографическую защиту, которая включает симметричные и несимметричные методы шифрования, формирование секретных ключей, протоколы ограничения доступа и аутентификации сообщений и пользователей. Кроме того, в пособии рассматриваются типовые уязвимости операционных и информационно-вычислительных систем.

\section*{Благодарности}
\addcontentsline{toc}{section}{Благодарности}
Авторы пособия благодарят студентов, аспирантов и сотрудников Московского физико-технического института (государственного университета), которые помогли с подготовкой, редактированием и поиском ошибок в тексте.

\setlength{\columnsep}{0.5em}
\begin{multicols}{2}
\begin{small}
\begin{itemize}[leftmargin=0.5em]\itemsep1pt \parskip0pt \parsep0pt

	\item[] Владимир Аверьянов\begin{tiny} (201-113 гр.)\end{tiny} %rfoofr
	\item[] Руслан Агишев\begin{tiny} (201-314 гр.)\end{tiny} %agruslan

	\item[] Алипаша Бабаев\begin{tiny} (201-311 гр.)\end{tiny} %alishnik
	\item[] Олег Бабин\begin{tiny} (201-415 гр.)\end{tiny} %olegrok
	\item[] Татьяна Бакланова\begin{tiny} (201-211 гр.)\end{tiny}
	\item[] Дмитрий Банков\begin{tiny} (201-011 гр.)\end{tiny}
	\item[] Александр Белов\begin{tiny} (201-214 гр.)\end{tiny}
	\item[] Даниил Бершацкий\begin{tiny} (201-012 гр.)\end{tiny}
	\item[] Анастасия Бодрова\begin{tiny} (201-218 гр.)\end{tiny} %AnastasiaBodrova
	\item[] Дмитрий Бородий\begin{tiny} (201-112 гр.)\end{tiny}
	\item[] Евгений Брицын\begin{tiny} (201-312 гр.)\end{tiny} %ebritsyn
	\item[] Олег Бусловский\begin{tiny} (201-219 гр.)\end{tiny} %oledzzka

	\item[] Вадим Варнавский\begin{tiny} (201-213 гр.)\end{tiny}
	\item[] Илья Васильев\begin{tiny} (201-217 гр.)\end{tiny}
	\item[] Эмиль Вахитов\begin{tiny} (201-114 гр.)\end{tiny}
	\item[] Дмитрий Вербицкий\begin{tiny} (201-119 гр.)\end{tiny}
	\item[] Константин Виноградов\begin{tiny} (201-114 гр.)\end{tiny} %Vikont133

	\item[] Тагир Гадельшин\begin{tiny} (201-119 гр.)\end{tiny}
	\item[] Марат Гаджибутаев\begin{tiny} (201-018 гр.)\end{tiny}
	\item[] Тимур Газизов\begin{tiny} (201-317 гр.)\end{tiny} %RiderOnTheStorm
	\item[] Ильназ Гараев\begin{tiny} (201-113 гр.)\end{tiny}
	\item[] Евгений Глушков\begin{tiny} (201-012 гр.)\end{tiny}
	\item[] Иван Голованов\begin{tiny} (201-312 гр.)\end{tiny} %ivangolovanov; legendawes, legendawes1
	\item[] Андрей Горбунов\begin{tiny} (201-116 гр.)\end{tiny} %goriand
	\item[] Елена Гундрова\begin{tiny} (201-214 гр.)\end{tiny}
	\item[] Алексей Гусаров\begin{tiny} (201-216 гр.)\end{tiny}
	\item[] Наталья Гусева\begin{tiny} (201-216 гр.)\end{tiny} %g-n-ev

	\item[] Андрей Диденко\begin{tiny} (201-311 гр.)\end{tiny} %DidenkoAndre
	\item[] Олег Дробот\begin{tiny} (201-317 гр.)\end{tiny} %tobord

	\item[] Дмитрий Ермилов\begin{tiny} (201-311 гр.)\end{tiny} %schnee-katze

	\item[] Сергей Жестков\begin{tiny} (201-013 гр.)\end{tiny}
	\item[] Андрей Житов\begin{tiny} (201-114 гр.)\end{tiny} %zhitHappens

	\item[] Виталий Занкин\begin{tiny} (201-111 гр.)\end{tiny} %zankin
	\item[] Дмитрий Зборовский\begin{tiny} (201-119 гр.)\end{tiny}

	\item[] Марат Ибрагимов\begin{tiny} (201-114 гр.)\end{tiny}
	\item[] Александр Иванов\begin{tiny} (201-011 гр.)\end{tiny}
	\item[] Александр Иванов\begin{tiny} (201-019 гр.)\end{tiny}
	\item[] Атнер Иванов\begin{tiny} (201-114 гр.)\end{tiny}
	\item[] Владимир Ивашкин\begin{tiny} (201-112 гр.)\end{tiny}

	\item[] Ирина Камалова\begin{tiny} (201-115 гр.)\end{tiny}
	\item[] Иван Киселёв\begin{tiny} (201-115 гр.)\end{tiny}
	\item[] Константин Ковальков\begin{tiny} (201-015 гр.)\end{tiny}
	\item[] Николай Козырский\begin{tiny} (201-417 гр.)\end{tiny} %NikolayKozyrskiy
	\item[] Федор Константинов\begin{tiny} (201-312 гр.)\end{tiny} %FRaKTT
	\item[] Анастасия Коробкина\begin{tiny} (201-312 гр.)\end{tiny} %korobkina
	\item[] Илья Копцов\begin{tiny} (201-115 гр.)\end{tiny} %IlyaKoptsov
	\item[] Андрей Кочетыгов\begin{tiny} (201-111 гр.)\end{tiny} %anko1774
	\item[] Сергей Кошечкин\begin{tiny} (201-213 гр.)\end{tiny}
	\item[] Александр Кравцов\begin{tiny} (201-116 гр.)\end{tiny}
	\item[] Анастасия Красавина\begin{tiny} (201-217 гр.)\end{tiny} %akrasavina
	\item[] Татьяна Красавина\begin{tiny} (201-214 гр.)\end{tiny} %tkrasav
	\item[] Виталий Крепак\begin{tiny} (201-013 гр.)\end{tiny}
	\item[] Егор Кривов\begin{tiny} (201-211 гр.)\end{tiny}
	\item[] Александр Кротов\begin{tiny} (201-011 гр.)\end{tiny}
	\item[] Ефим Крохин\begin{tiny} (201-217 гр.)\end{tiny} %Yefim-Krokhin
	\item[] Станислав Круглик\begin{tiny} (201-111 гр.)\end{tiny}
	\item[] Павел Крюков\begin{tiny} (200-916 гр.)\end{tiny} %pavelkryukov
	\item[] Аркадий Кудашов\begin{tiny} (201-317 гр.)\end{tiny} %ark85
	\item[] Денис Кудяков\begin{tiny} (201-314 гр.)\end{tiny} %Denis7775
	\item[] Егор Кузнецов\begin{tiny} (201-211 гр.)\end{tiny}
	\item[] Зулкаид Курбанов\begin{tiny} (201-113 гр.)\end{tiny}

	\item[] Всеволод Ливинский\begin{tiny} (201-216 гр.)\end{tiny} %Vsevolod-Livinskij
	\item[] Артемий Лузянин\begin{tiny} (201-312 гр.)\end{tiny} %artemluzyanin

	\item[] Егор Макарычев\begin{tiny} (201-115 гр.)\end{tiny}
	\item[] Иван Макеев\begin{tiny} (201-212 гр.)\end{tiny} %katala777
	\item[] Ольга Малюгина\begin{tiny} (201-111 гр.)\end{tiny}
	\item[] Алексей Мамаков\begin{tiny} (201-113 гр.)\end{tiny} %AlekseyMamakov
	\item[] Роман Маракулин\begin{tiny} (201-211 гр.)\end{tiny}
	\item[] Андрей Мартыненко\begin{tiny} (201-312 гр.)\end{tiny}
	\item[] Александр Матков\begin{tiny} (201-314 гр.)\end{tiny} %alexmatkov
	\item[] Артём Меринов\begin{tiny} (201-214 гр.)\end{tiny}
	\item[] Даниил Меркулов\begin{tiny} (201-111 гр.)\end{tiny}
	\item[] Олег Милосердов\begin{tiny} (201-016 гр.)\end{tiny}
	\item[] Дао Куанг Минь\begin{tiny} (201-116 гр.)\end{tiny}
	\item[] Антон Митрохин\begin{tiny} (201-216 гр.)\end{tiny} %ncos
	\item[] Надежда Мозолина\begin{tiny} (201-119 гр.)\end{tiny}
	\item[] Дарья Мороз\begin{tiny} (201-318 гр.)\end{tiny} %moryshka

	\item[] Хыу Чунг Нгуен\begin{tiny} (201-015 гр.)\end{tiny} %huutrung
	\item[] Артём Никитин\begin{tiny} (201-012 гр.)\end{tiny}
	\item[] Евгения Никольская\begin{tiny} (201-115 гр.)\end{tiny} %EvgeniyaNikolskaya

	\item[] Александр Ометов\begin{tiny} (201-113 гр.)\end{tiny} %Ozzmorn
	\item[] Даниил Охлопков\begin{tiny} (201-311 гр.)\end{tiny}

	\item[] Александр Парамонов\begin{tiny} (201-416 гр.)\end{tiny} %AleksandrParamonov
	\item[] Дмитрий Паршин\begin{tiny} (201-313 гр.)\end{tiny} %dmitryparshin
	\item[] Даниил Похачевский\begin{tiny} (201-519 гр.)\end{tiny} %(no github account, via VK)
	\item[] Роман Проскин\begin{tiny} (201-316 гр.)\end{tiny} %opomuc
	\item[] Андрей Пунь\begin{tiny} (201-013 гр.)\end{tiny}

	\item[] Дмитрий Радкевич\begin{tiny} (201-316 гр.)\end{tiny} %radkevichdmitriy
	\item[] Артём Рудой\begin{tiny} (201-211 гр.)\end{tiny} %artemrudoj
	\item[] Сергей Рудаков\begin{tiny} (201-219 гр.)\end{tiny} %Shytnichok

	\item[] Вадим Сафронов\begin{tiny} (201-112 гр.)\end{tiny}
	\item[] Евгения Сахно\begin{tiny} (201-317 гр.)\end{tiny} %EvgenyaSakhno
	\item[] Иван Саюшев\begin{tiny} (201-112 гр.)\end{tiny}
	\item[] Александр Сергеев\begin{tiny} (201-318 гр.)\end{tiny} %sanekas
	\item[] Всеволод Сергеев\begin{tiny} (201-212 гр.)\end{tiny} %VsevolodSergeev
	\item[] Григорий Соболь\begin{tiny} (201-316 гр.)\end{tiny} %grishasobol
	\item[] Иван Соколов\begin{tiny} (201-314 гр.)\end{tiny} %vansokol
	\item[] Илья Соломатин\begin{tiny} (201-211 гр.)\end{tiny}
	\item[] Игорь Сорокин\begin{tiny} (201-112 гр.)\end{tiny}
	\item[] Вера Сосновик\begin{tiny} (201-214 гр.)\end{tiny} %Sosnovik
	\item[] Игорь Степанов\begin{tiny} (201-213 гр.)\end{tiny}
	\item[] Мария Столяренко\begin{tiny} (201-214 гр.)\end{tiny}
	\item[] Светлана Субботина\begin{tiny} (201-316 гр.)\end{tiny} %sonne122
	\item[] Виктор Сухарев\begin{tiny} (201-114 гр.)\end{tiny} %ViktorSuharev

	\item[] Буй Зуи Тан\begin{tiny} (201-112 гр.)\end{tiny}
	\item[] Михаил Тверье\begin{tiny} (201-313 гр.)\end{tiny} %MishaTvr
	\item[] Тимофей Тормагов\begin{tiny} (201-316 гр.)\end{tiny} %tormagov
	\item[] Артём Тучин\begin{tiny} (201-217 гр.)\end{tiny} %tuchart
	\item[] Татьяна Тюпина\begin{tiny} (201-116 гр.)\end{tiny}

	\item[] Сергей Угрюмов\begin{tiny} (201-119 гр.)\end{tiny}
	\item[] Илья Улитин\begin{tiny} (201-417 гр.)\end{tiny} %Mobilnik

	\item[] Марсель Файзуллин\begin{tiny} (201-114 гр.)\end{tiny}
	\item[] Нияз Фазлыев\begin{tiny} (201-114 гр.)\end{tiny} %NiyazFazliev
	\item[] Айдар Фасхутдинов\begin{tiny} (201-114 гр.)\end{tiny} %aidarfaskh
	\item[] Наталья Федотова\begin{tiny} (201-212 гр.)\end{tiny}
	\item[] Данил Филиппов\begin{tiny} (201-115 гр.)\end{tiny}
	\item[] Яков Фиронов\begin{tiny} (201-314 гр.)\end{tiny} %YakovFironov

	\item[] Никита Харичкин\begin{tiny} (201-315 гр.)\end{tiny} %NEKharichkin
	\item[] Тарас Хахулин\begin{tiny} (201-417 гр.)\end{tiny} %Khakhulin
	\item[] Алексей Хацкевич\begin{tiny} (201-211 гр.)\end{tiny}

	\item[] Александра Цветкова\begin{tiny} (201-216 гр.)\end{tiny}

	\item[] Андрей Шишпанов\begin{tiny} (201-316 гр.)\end{tiny} %Shispan

	\item[] Евгений Юлюгин\begin{tiny} (201-916 гр.)\end{tiny}
	\item[] Руслан Юсупов\begin{tiny} (201-211 гр.)\end{tiny}
\end{itemize}
\end{small}
\end{multicols}

\subimport*{history/}{index}

\section{Основные понятия}
\selectlanguage{russian}
Для успешного выполнения любых целей по защите информации необходимо участие в процессе защиты нескольких субъектов, которые по определённым правилам будут выполнять технические или организационные действия, криптографические операции, взаимодействовать друг с другом, например, передавая сообщения или проверяя личности друг друга.

Формализация подобных действий делается через описание протокола. \emph{Протокол} -- описание распределённого алгоритма, в процессе выполнения которого два или более участников последовательно выполняют определённые действия и обмениваются сообщениями\footnote{Здесь и далее в этом разделе определения даны на основе~\cite{Cheremushkin:2009}.}.

Под участником\index{участник!протокола} (субъектом\index{субъект!протокола}, стороной\index{сторона!протокола}) протокола понимают не только людей, но и приложения, группы людей или целые организации. Формально участниками считают только тех, кто выполняет активную роль в рамках протокола. Хотя при создании и описании протоколов забывать про пассивные стороны тоже не стоит. Например, пассивный криптоаналитик\index{криптоаналитик!пассивный} формально не является участником протоколов, но многие протоколы разрабатываются с учётом защиты от таких <<неучастников>>.

Протокол состоит из \emph{циклов}\index{цикл!протокола} (\langen{round}) или \emph{проходов}\index{проход!протокола} (\langen{pass}). Цикл -- временной интервал активности только одного участника. За исключением самого первого цикла протокола, обычно начинается приёмом сообщения, а заканчивается -- отправкой.

Цикл (или проход) состоит из \emph{шагов} (действий, \langen{step, action}) -- конкретных законченных действий, выполняемых участником протокола. Например:
\begin{itemize}
	\item генерация нового (случайного) значения;
	\item вычисление значений функции;
	\item проверка сертификатов, ключей, подписей, и др.;
	\item приём и отправка сообщений.
\end{itemize}

Прошедшая в прошлом или даже просто теоретически описанная реализация протокола для конкретных участников называется \emph{сеансом}\index{сеанс!протокола}. Каждый участник в рамках сеанса выполняет одну или несколько \emph{ролей}. В другом сеансе протокола участники могут поменяться ролями и выполнять уже совсем другие функции.

Можно сказать, что протокол прескрептивно описывает правила поведения каждой роли в протоколе. А сеанс это дескриптивное описание (возможно теоретически) состоявшейся в прошлом реализации протокола.

Пример описания протокола.
\begin{enumerate}
	\item Участник с ролью <<Отправитель>> должен отправить участнику с ролью <<Получатель>> сообщение.
	\item Участник с ролью <<Получатель>> должен принять от участника с ролью <<Отправитель>> сообщение.
\end{enumerate}

Пример описания сеанса протокола.
\begin{enumerate}
	\item 1-го апреля в 13:00 Алиса отправила Бобу сообщение.
	\item 1-го апреля в 13:05 Боб принял от Алисы сообщение.
\end{enumerate}

\emph{Защищённым протоколом}\index{протокол!защищённый} или \emph{протоколом обеспечения безопасности}\index{протокол!обеспечения безопасности} будет называть протокол, обеспечивающий выполнение хотя бы одной защитной функции~\cite{ISO:7498-2:1989}:
\begin{itemize}
	\item аутентификация сторон и источника данных,
	\item разграничение доступа,
	\item конфиденциальность,
	\item целостность,
	\item невозможность отказа от факта отправки или получения.
\end{itemize}

Если защищённый протокол предназначен для выполнения функций безопасности криптографической системы, или если в процессе его исполнения используются криптографические алгоритмы, то такой протокол будем называть \emph{криптографическим}\index{протокол!криптографический}.


\chapter{Классические шифры}

В главе приведены наиболее известные \emph{классические} шифры, которыми можно было пользоваться до появления роторных машин. К ним относятся такие шифры, как шифр Цезаря\index{шифр!Цезаря}, шифр Плейфера\index{шифр!Плейфера}, шифр Хилла\index{шифр!Хилла} и шифр Виженера\index{шифр!Виженера}. Они наглядно демонстрируют различные классы шифров.

\input{monoalphabetic_ciphers}

\section{Биграммные шифры замены}\index{шифр!биграммный}
\selectlanguage{russian}

Если при шифровании преобразуется по две буквы открытого текста, то такой шифр называется \emph{биграммным}\index{шифр!биграммный} шифром замены. Первый биграммный шифр был изобретён аббатом Иоганном Тритемием и опубликован в 1508-м году. Другой биграммный шифр изобретён в 1854 году Чарльзом Витстоном. Лорд Лайон Плейфер (\langen{Lyon Playfair}) внедрил этот шифр в государственных службах Великобритании, и шифр был назван шифром Плейфера\index{шифр!Плейфера}.

Опишем шифр Плейфера\index{шифр!Плейфера}. Составляется таблица для английского алфавита (буквы \texttt{I}, \texttt{J} отождествляются), в которую заносятся буквы перемешанного алфавита, например, в виде таблицы, представленной ниже. Часто перемешивание алфавита реализуется с помощью начального слова, в котором отбрасываются повторяющиеся символы. В нашем примере начальное слово \texttt{playfair}. Таблица имеет вид:

\begin{center}
    \begin{tabular}{ccccc}
        p & l & a & y & f  \\
        i & r & b & c & d  \\
        e & g & h & k & m  \\
        n & o & q & s & t  \\
        u & v & w & x & z  \\
    \end{tabular}
\end{center}

Буквы открытого текста разбиваются на пары. Правила шифрования каждой пары состоят в следующем.

\begin{itemize}
    \item Если буквы пары не лежат в одной строке или в одном столбце таблицы, то они заменяются буквами, образующими с исходными буквами вершины прямоугольника. Первой букве пары соответствует буква таблицы, находящаяся в том же столбце. Пара букв открытого текста \texttt{we} заменяется двумя буквами таблицы \texttt{hu}. Пара букв открытого текста \texttt{ew} заменяется двумя буквами таблицы \texttt{uh}.
    \item Если буквы пары открытого текста расположены в одной строке таблицы, то каждая буква заменяется соседней справа буквой таблицы. Например, пара \texttt{gk} заменяется двумя буквами \texttt{hm}. Если одна из этих букв -- крайняя правая в таблице, то её <<правым соседом>> считается крайняя левая в этой строке. Так, пара \texttt{to} заменяется буквами \texttt{nq}.
    \item Если буквы пары лежат в одном столбце, то каждая буква заменяется соседней буквой снизу. Например, пара \texttt{lo} заменяется парой \texttt{rv}. Если одна из этих букв крайняя нижняя, то её <<нижним соседом>> считается крайняя верхняя буква в этом столбце таблицы. Например, пара \texttt{kx} заменяется буквами \texttt{sy}.
    \item Если буквы в паре одинаковые, то между ними вставляется определённая буква, называемая <<буквой-пустышкой>>. После этого разбиение на пары производится заново.
\end{itemize}

\example
Используем шифр Плейфера\index{шифр!Плейфера} и зашифруем сообщение "\texttt{Wheatstone was the inventor}". Исходное сообщение, разбитое на биграммы, показано в первой строке таблицы. Результат шифрования, также разбитый на биграммы, приведён во второй строке.
\begin{center} \begin{tabular}{|*{12}c|}
    \hline
    wh & ea & ts & to & ne & wa & st & he & in & ve & nt & or \\
    \hline
    aq & ph & nt & nq & un & ab & tn & kg & eu & gu & on & vg \\
    \hline
\end{tabular} \end{center}
\exampleend

Шифр Плейфера\index{шифр!Плейфера} не является криптографически стойким. Несложно найти ключ, если известны пара открытого текста и соответствующего ему шифртекста. Если известен только шифртекст, криптоаналитик может проанализировать соответствие между частотой появления биграмм в шифртексте и известной частотой появления биграмм в языке, на котором написано сообщение. Такой частотный анализ помогает дешифрованию.


\input{hills_cipher}

% \subsection{Омофонные замены}
%
% Омофонными заменами называют криптопримитивы, в основе которых лежит замена групп символов открытого текста $M$ на группу символов $C$ с использованием ключа $K$. Такой метод шифрования вносит неоднозначность между $M$ и $C$, это позволяет защититься от методов частотного криптоанализа.
%  \subsection{шифрокоды}
%  Шифрокоды -- это класс шифров сочетающих в себе свойства кодов и помехозащищённости со свойствами шифра и обеспечения конфиденциальности.

\input{vigeneres_cipher}

\input{polyalphabetic_cipher_cryptanalysis}

\input{perfect_secure_systems}

\subimport*{block-ciphers/}{index}

\chapter{Генераторы псевдослучайных чисел}\label{chapter-generators}
\selectlanguage{russian}

Для работы многих криптографических примитивов необходимо уметь получать случайные числа:
\begin{itemize}
	\item вектор инициализации для отдельных режимов сцепления блоков должен быть случайным числом (см. раздел~\ref{section-block-chaining});
	\item для генерации пар открытых и закрытых ключей необходимы случайные числа (см. главу~\ref{chapter-public-key});
	\item стойкость многих криптографических протоколов ключей (см. главу~\ref{chapter-protocols}) основывается в том числе на выработке случайных чисел (\langen{nonce}), которые не может предугадать злоумышленник.
\end{itemize}

Генератором случайных чисел (\langen{random number generator})\index{генератор!случайных чисел} мы будем называть процесс\footnote{Есть и строгое математическое определение генератора в общем смысле. Генератором называется функция $g: \left\{0, 1\right\}^{n} \to \left\{0, 1\right\}^{q\left(n\right)}$, вычислимая за полиномиальное время. Однако мы пока не будем использовать это определение, чтобы показать разницу между истинно случайными числами и псевдослучайными.}, результатом работы которого является случайная последовательность чисел, а именно такая, что зная произвольное число предыдущих чисел последовательности (и способ их получения), даже теоретически нельзя предсказать следующее с вероятностью больше заданной. К таким случайным процессам можно отнести:

\begin{itemize}
	\item результат работы счётчика элементарных частиц, работа с которым включена в лабораторный практикум по общей физике для студентов первого курса МФТИ;
	\item время между нажатиями клавиш на клавиатуре персонального компьютера или расстояние, которое проходит <<мышь>> во время движения;
	\item время между двумя пакетами, полученными сетевой картой;
	\item тепловой шум, измеряемый звуковой картой на входе аналогового микрофона, даже при отсутствии самого микрофона.
\end{itemize}

Хотя для всех этих процессов можно предсказать приблизительное значение (чётное или нечётное), его последний бит будет оставаться достаточно случайным для практических целей. С учётом данной поправки их можно называть надёжными или качественными генераторами случайных чисел.

Однако к генератору случайных чисел предъявляются и другие требования. Кроме уже указанного критерия \emph{качественности} или \emph{надёжности}, генератор должен быть \emph{быстрым} и \emph{дешёвым}. Быстрым -- чтобы получить большой объём случайной информации за заданный период времени. И дешёвым -- чтобы его можно было бы использовать на практике. Количество случайной информации от перечисленных выше генераторов составляет не более десятков килобайт в секунду (для теплового шума) и значительно меньше, если мы будем требовать ещё и равномерность распределения полученных случайных чисел.

С целью получения большего объёма случайной информации используют специальные алгоритмы, которые называют генераторами псевдослучайных чисел (ГПСЧ). ГПСЧ -- это детерминированный алгоритм, выходом которого является последовательность чисел, обладающая свойством случайности. Работу ГПСЧ можно описать следующей моделью. На подготовительном этапе оперативная память, используемая алгоритмом, заполняется начальным значением (\langen{seed}). Далее на каждой итерации своей работы ГПСЧ выдаёт на выход число, которое является функцией от состояния оперативной памяти алгоритма, и меняет содержимое своей памяти по определённым правилам. Содержимое оперативной памяти называется \emph{внутренним состоянием} генератора.

Как и у любого алгоритма, у ГПСЧ есть определённый размер используемой оперативной памяти\footnote{Только алгоритмы с фиксированным размером используемой оперативной памяти и можно называть \emph{генераторами} в строгом математическом смысле этого слова, как следует из определения.}. Исходя из практических требований, предполагается, что размер оперативной памяти для ГПСЧ сильно ограничен. Так как память алгоритма ограничена, то ограничено и число различных внутренних состояний алгоритма. В силу того, что выдаваемые ГПСЧ числа являются функцией от внутреннего состояния, то любой ГПСЧ, работающий с ограниченным размером оперативной памяти и не принимающий извне дополнительной информации, будет иметь \emph{период}. Для генератора с памятью в $n$ бит максимальный период, очевидно, равен $2^n$.

Качество детерминированного алгоритма, то есть то, насколько полученная последовательность обладает свойством случайной, можно оценить с помощью тестов, таких как набор тестов NIST (\langen{National Institute of Standards and Technology}, США,~\cite{NIST:2001}). Данный набор содержит большое число различных проверок, включая частотные тесты бит и блоков, тесты максимальных последовательностей в блоке, тесты матриц и так далее.

\input{linear-congruential-generator}

\input{lfsr}

\input{crypto-random}

\input{bbs_generator}

\section{КСГПСЧ на основе РСЛОС}

Как уже упоминалось ранее, использование РСЛОС в качестве ГПСЧ не является криптографически стойким. Однако можно использовать комбинацию из нескольких регистров сдвига, чтобы в результате получить быстрый, простой (дешёвый) и надёжный (криптографически стойкий) генератор псевдослучайных чисел.

\input{generators_with_multiple_shift_registers}

\input{generators_with_nonlinear_transformations}

\input{majority_generators}


\input{stream-ciphers}

\subimport*{hash-functions/}{index}

\input{public-key}

\subimport*{protocols/}{index}

\subimport*{secret-sharing/}{index}

\subimport*{secure-systems-examples/}{index}

\chapter{Аутентификация пользователя}


\section{Многофакторная аутентификация}

Для защищённых приложений применяется \emph{многофакторная} аутентификация одновременно по факторам различной природы:
\begin{enumerate}
    \item Свойство, которым обладает субъект. Например: биометрия, природные уникальные отличия (лицо, радужная оболочка глаз, папиллярные узоры, последовательность ДНК).
    \item Знание -- информация, которую знает субъект. Например: пароль, PIN (Personal Identification Number).
    \item Владение -- вещь, которой обладает субъект. Например: электронная или магнитная карта, флэш-память.
%    \item Факторы присвоения. Например, номер машины, RFID-метка.
\end{enumerate}

В обычных массовых приложениях из-за удобства использования применяется аутентификация только по \emph{паролю}\index{пароль}, который является общим секретом пользователя и информационной системы. Биометрическая аутентификация по отпечаткам пальцев применяется существенно реже. Как правило, аутентификация по отпечаткам пальцев является дополнительным, а не вторым обязательным фактором (тоже из-за удобства её использования).

%Так же явно или неявно используется аутентификация по факторам:
%\begin{enumerate}
%    \item Социальная сеть. Доверие к индивидууму в личном или интернет общении, на основании общих связей.
%    \item Географическое положение. Например, для проверки оплаты товаров по кредитной карте.
%    \item Время. Доступ к сервисам или местам только в определённое время.
%    \item И др.
%\end{enumerate}


\section[Энтропия и криптостойкость паролей]{Энтропия и криптостойкость \protect\\ паролей}

Стандартный набор символов паролей, которые можно набрать на клавиатуре, используя английские буквы и небуквенные символы, состоит из $D=94$ символов. При длине пароля $L$ символов и предположении равновероятного использования символов энтропия паролей равна
    \[ H = L \log_2 D. \]

Клод Шеннон, исследуя энтропию символов английского текста, изучал вероятность успешного предсказания людьми следующего символа по первым нескольким символам слов или текста. В результате Шеннон получил оценку энтропии первого символа $s_1$ текста порядка $H(s_1) \approx 4{,}6$--$4{,}7$ бит/символ и оценки энтропий последующих символов, постепенно уменьшающиеся до $H(s_9) \approx 1{,}5$ бит/символ для 9-го символа. Энтропия для длинных текстов литературных произведений получила оценку $H(s_\infty) \approx 0{,}4$ бит/символ.

Статистические исследования баз паролей показывают, что наиболее часто используются буквы <<a>>, <<e>>, <<o>>, <<r>> и цифра <<1>>.

NIST (Национальный институт стандартов и технологий США, \langen{National Institute of Standards and Technology})  использует следующие рекомендации для оценки энтропии паролей\index{энтропия!пароля}, создаваемых людьми.
\begin{enumerate}
    \item Энтропия первого символа $H(s_1) = 4$ бит/символ.
    \item Энтропия со 2-го по 8-й символы $H(s_{i}) = 2$ бит/символ, $2 \leq i \leq 8$.
    \item Энтропия с 9-го по 20-й символы $H(s_{i}) = 1{,}5$ бит/символ, $9 \leq i \leq 20$.
    \item Энтропия с 21-го символа $H(s_{i}) = 1$ бит/символ, $i \geq 21$.
    \item Проверка композиции на использование символов разных регистров и небуквенных символов добавляет до 6-ти бит энтропии пароля.
    \item Словарная проверка на слова и часто используемые пароли добавляет до 6 бит энтропии для коротких паролей. Для 20-символьных и более длинных паролей прибавка к энтропии -- 0 бит.
\end{enumerate}

Для оценки энтропии пароля нужно сложить энтропии символов $H(s_i)$ и сделать дополнительные надбавки, если пароль удовлетворяет тестам на композицию и отсутствует в словаре.

\begin{table}[!ht]
    \caption{Оценка NIST предполагаемой энтропии паролей\label{tab:password-entropy}}
    \resizebox{\textwidth}{!}{ \begin{tabular}{|c||c|c|c||c|}
        \hline
        \multirow{2}{*}{\parbox{1.5cm}{\medskip \centering Длина пароля, символы}} & \multicolumn{3}{|c||}{\parbox{6cm}{\centering Энтропия паролей пользователей по критериям NIST}} & \multirow{2}{*}{\parbox{3cm}{\centering Энтропия случайных равновероятных паролей}} \\
        \cline{2-4}
        & \parbox{1.5cm}{\centering Без проверок} & \parbox{2cm}{\centering Словарная проверка} & \parbox{3cm}{\centering Словарная и композиционная проверка} & \\
        \hline
        4  & 10 & 14 & 16 & 26.3 \\
        6  & 14 & 20 & 23 & 39.5 \\
        8  & 18 & 24 & 30 & 52.7 \\
        10 & 21 & 26 & 32 & 65.9 \\
        12 & 24 & 28 & 34 & 79.0 \\
        16 & 30 & 32 & 38 & 105.4 \\
        20 & 36 & 36 & 42 & 131.7 \\
        24 & 40 & 40 & 46 & 158.0 \\
        30 & 46 & 46 & 52 & 197.2 \\
        40 & 56 & 56 & 62 & 263.4 \\
        \hline
    \end{tabular} }
\end{table}

В таблице~\ref{tab:password-entropy} приведена оценка NIST на величину энтропии пользовательских паролей в зависимости от их длины, и приведено сравнение с энтропией случайных паролей с равномерным распределением символов из набора в $D=94$ символов клавиатуры. Вероятное число попыток для подбора пароля составляет $O(2^H)$. Из таблицы видно, что по критериям NIST энтропия реальных паролей в 2--4 раза меньше энтропии случайных паролей с равномерным распределением символов.

\example
Оценим общее количество существующих паролей. Население Земли -- 7 млрд. Предположим, что всё население использует компьютеры и Интернет, и у каждого человека по 10 паролей. Общее количество существующих паролей -- $7 \cdot 10^{10} \approx 2^{36}$.

Имея доступ к наиболее массовым интернет-сервисам с количеством пользователей десятки и сотни миллионов, в которых пароли часто хранятся в открытом виде из-за необходимости обновления ПО и, в частности, выполнения аутентификации, мы:
\begin{enumerate}
	\item имеем базу паролей, покрывающую существенную часть пользователей; 
	\item можем статистически построить правила генерирования паролей.
\end{enumerate}

Даже если пароль хранится в защищённом виде, то при вводе пароль, как правило, в открытом виде пересылается по Интернету, и все преобразования пароля для аутентификации осуществляет интернет-сервис, а не веб-браузер. Следовательно, интернет-сервис имеет доступ к исходному паролю.
\exampleend

В 2002 г. был подобран ключ для 64-битного блочного шифра RC5 сетью персональных компьютеров \texttt{distributed.net}, выполнявших вычисления в фоновом режиме. Суммарное время вычислений всех компьютеров -- 1757 дней, было проверено 83\% пространства всех ключей. Это означает, что пароли с оценочной энтропией менее 64 бит, то есть \emph{все пароли} до 40 символов по критериям NIST, могут быть подобраны в настоящее время. Конечно, с оговорками на то, что 1) нет ограничений на количество и частоту попыток аутентификаций, 2) алгоритм генерации вероятных паролей эффективен.

Строго говоря, использование даже 40-символьного пароля для аутентификации или в качестве ключа блочного шифрования является небезопасным.


\subsubsection{Число паролей}

Приведём различные оценки числа паролей, создаваемых людьми. Чаще всего такие пароли основаны на словах или закономерностях естественного языка. В английском языке всего около $1\ 000\ 000 \approx 2^{20}$ слов, включая термины.

%http://www.springerlink.com/content/bh216312577r6w64/fulltext.pdf
%http://www.antimoon.com/forum/2004/4797.htm

Используемые слоги английского языка имеют вид V, CV, VC, CVV, VCC, CVC, CCV, CVCC, CVCCC, CCVCC, CCCVCC, где C -- согласная (consonant), V -- гласная (vowel). 70\% слогов имеют структуру VC или CVC. Общее число слогов $S = 8000 \dots 12000$. Средняя длина слога -- 3 буквы.

Предполагая равновероятное распределение всех слогов английского языка, для числа паролей из $r$ слогов получим верхнюю оценку
    \[ N_1 = S^r = 2^{13 r} \approx 2^{4.3 L_1}. \]
Средняя длина паролей составит:
    \[ L_1 \approx 3 r. \]

Теперь предположим, что пароли могут состоять только из 2--3 буквенных слогов вида CV, VC, CVV, VCC, CVC, CCV с равновероятным распределением символов. Подсчитаем число паролей $N_2$, которые могут быть построены из $r$ таких слогов. В английском алфавите число гласных букв $n_v = 10$, согласных $n_c = 16$, $n = n_v + n_c = 26$. Верхняя оценка числа $r$-слоговых паролей:
    \[ N_2 = (n_c n_v + n_v n_c + n_c n_v n_v + n_v n_c n_c + n_c n_v n_c + n_c n_c n_v)^r \approx \]
        \[ \approx \left( n_c n_v(3 n_c + n_v) \right)^r, \]
    \[ N_2 \approx \left( \frac{n^3}{2} \right)^r \approx 2^{13 r} \approx 2^{4.3 L_2}. \]
Средняя длина паролей:
    \[ L_2 = \frac{n_c n_v(2 + 2 + 3 n_v + 3 n_c + 3 n_c + 3 n_c)}{n_c n_v (1 + 1 + n_v + n_c + n_c + n_c)} \cdot r \approx 3 r. \]

Как видно, в обоих предположениях получились одинаковые оценки для числа и длины паролей.

Подсчитаем верхние оценки числа паролей из $L$ символов, предполагая равномерное распределение символов из алфавита мощностью $D$ символов: a) $D_1 = 26$ строчных букв, б) все $D_2 = 94$ печатных символа клавиатуры (латиница и небуквенные символы):
    \[ N_3 = D_1^L \approx 2^{4.7 L}, \]
    \[ N_4 = D_2^L \approx 2^{6.6 L}. \]

\begin{table}[!ht]
    \caption{Различные верхние оценки числа паролей\label{tab:password-number}}
    \resizebox{\textwidth}{!}{ \begin{tabular}{|c||c|c|c|}
        \hline
        \multirow{2}{*}{\parbox{1.5cm}{\medskip\medspace \centering Длина пароля}} & \multicolumn{3}{|c|}{Число паролей} \\
        \cline{2-4}
            & \parbox{3.5cm}{\medspace \centering На основе слоговой композиции} &
            \parbox{3cm}{\medspace\centering Алфавит $D=26$ символов} &
            \parbox{3cm}{\medspace \centering Алфавит $D=94$ символа} \\
        \hline
        \rule{0pt}{2.5ex}$6$  & $2^{26}$ & $2^{28}$ & $2^{39}$ \\
        9  & $2^{39}$ & $2^{42}$ & $2^{59}$ \\
        12 & $2^{52}$ & $2^{56}$ & $2^{79}$ \\
        15 & $2^{65}$ & $2^{71}$ & $2^{98}$ \\
        \hline
        \rule{0pt}{2.5ex} 21 & $2^{91}$ & $2^{99}$ & $2^{137}$ \\
        \hline
        \rule{0pt}{2.5ex} 39 & $2^{169}$ & $2^{183}$ & $2^{256}$ \\
        \hline
    \end{tabular} }
\end{table}

Из таблицы~\ref{tab:password-number} видно, что при доступном объёме вычислений в $2^{60}$\,--\,$2^{70}$ операций, пароли вплоть до 15-ти символов, построенные на словах, слогах, изменениях слов, вставках цифр, небольшом изменении регистров и других простейших модификациях, в настоящее время могут быть найдены полным перебором как на вычислительном кластере, так и на персональном компьютере.

Для достижения криптостойкости паролей, сравнимой со 128- или 256-битовым секретным ключом, требуется выбирать пароль из 20 и 40 символов соответственно, что, как правило, не реализуется из-за сложности запоминания и возможных ошибок при вводе.


%Подсчитаем число паролей $N_1$, которые могут могут построены из $r$ ~ 2-3 буквенных слогов: $cv, vc, ccv, cvc, vcc$, где $c$ -- согласная, $v$ -- гласная. В английском алфавите $n_v = 10, n_c = 16, n = n_v + n_c = 26$. Число паролей
%    \[ N_1 = \left( n_v n_c (1 + 1 + n_c + n_c + n_c) \right)^r \approx 3^r n_v^r n_c^{2r}. \]
%Средняя длина паролей
%    \[ L = r \left( \frac{2 + 2 + 3 n_c + 3 n_c + 3 n_c}{1 + 1 + n_c + n_c + n_c} \right) \approx 3r. \]
%
%%Учтем, что $b \leq r$ символов могут быть заглавными: $N_1 \rightarrow N_2 < N_1 \binom{L}{b} \left( \frac{n}{n_v} \right)^b$. Вставим $d$ цифр в случайные места: $N_2 \rightarrow N_3 = N_2 (10 (1 + L))^d \approx N_2 (10 L)^d$.
%%
%%Общее число паролей
%%    \[ N = N_3 = 3^r 10^r 16^{2r} \binom{3r}{b} 2.6^b \left(10 \cdot 3 r \right)^d. \]
%%
%%\begin{table}[!ht]
%%    \centering
%%    \small
%%    \begin{tabular}{|c|c|c|c|c||cr|}
%%        \hline
%%        \parbox{1.3cm}{Слогов, $r$} & \parbox{1.8cm}{Заглавных букв, $b$} & \parbox{1.5cm}{Вставок цифр, $d$} & \parbox{2.8cm}{Средняя длина пароля, $L+d$} & \parbox{3cm}{Верхняя оценка числа паролей $N$} & \multicolumn{2}{|c|}{\parbox{3.2cm}{Число всех паролей}} \\
%%        \hline
%%        $2$ & $0$ & $0$ & $6$ & $2^{26}$ & $2^{36}$ & a-z \\
%%        $2$ & $2$ & $0$ & $6$ & $2^{32}$ & $2^{48}$ & A-Z, a-z \\
%%        $2$ & $2$ & $2$ & $8$ & $2^{45}$ & $2^{48}$ & A-Z, a-z, 0-9 \\
%%        \hline
%%        $3$ & $0$ & $0$ & $9$ & $2^{39}$ & $2^{54}$ & a-z \\
%%        $3$ & $3$ & $0$ & $9$ & $2^{49}$ & $2^{54}$ & A-Z, a-z \\
%%        $3$ & $3$ & $2$ & $11$ & $2^{63}$ & $2^{65}$ & A-Z, a-z, 0-9 \\
%%        \hline
%%        $4$ & $0$ & $0$ & $12$ & $2^{52}$ & $2^{93}$ & a-z \\
%%        $4$ & $3$ & $0$ & $12$ & $2^{64}$ & $2^{186}$ & A-Z, a-z \\
%%        $4$ & $3$ & $2$ & $14$ & $2^{78}$ & $2^{222}$ & A-Z, a-z, 0-9 \\
%%        \hline
%%    \end{tabular}
%%    \caption{Сравнение верхней оценки числа паролей, построенных на слогах, со всем доступным множеством паролей.}
%%    \label{tab:password-number}
%%\end{table}
%
%Учтем, что $b$ символов в пароле могут быть взяты не из 26-символьного алфавита строчных букв, а из всего алфавита в $D=94$ печатных символа клавиатуры (латиница и небуквенные символы):
%\[
%    \begin{array}{ll}
%    b=1 & N_1 \rightarrow N_2 = \frac{n_v}{n_v+n_c} 3^r n_v^{r-1} n_c^{2r} \cdot L. \]
%
%    \[ N_1 \rightarrow N_2 < N_1 \binom{L}{b} \left( \frac{D}{n_v} \right)^b. \]
%
%
%
%Общее число паролей
%    \[ N < 3^r n_v^r n_c^{2r} \binom{L}{b} \left( \frac{D}{n_v} \right)^b = 3^r 10^r 16^{2r} \binom{3r}{b} \left( \frac{94}{10} \right)^b. \]
%
%\begin{table}[!ht]
%    \centering
%    \small
%    \begin{tabular}{|c|c|c|c||cr|}
%        \hline
%        \parbox{1.5cm}{Слогов, $r$} & \parbox{3cm}{Средняя длина пароля, $L$} & \parbox{3cm}{Символов из всего алфавита, $b$} & \parbox{3cm}{Верхняя оценка числа паролей $N$} & \multicolumn{2}{|c|}{\parbox{3.2cm}{Число всех паролей, $D^L$}} \\
%        \hline
%        \multirow{3}{*}{2} & \multirow{3}{*}{6} & $0$ & $2^{26}$ & $2^{28}$ & a-z \\
%        & & $1$ & $2^{32}$ & $2^{34}$ & A-Z, a-z \\
%        & & $3$ & $2^{40}$ & $2^{39}$ & Весь алфавит \\
%        \hline
%        \multirow{3}{*}{3} & \multirow{3}{*}{9} & $0$ & $2^{39}$ & $2^{42}$ & a-z \\
%        & & $2$ & $2^{50}$ & $2^{51}$ & A-Z, a-z \\
%        & & $4$ & $2^{59}$ & $2^{59}$ & Весь алфавит \\
%        \hline
%        \multirow{3}{*}{4} & \multirow{3}{*}{12} & $0$ & $2^{52}$ & $2^{56}$ & a-z \\
%        & & $3$ & $2^{69}$ & $2^{68}$ & A-Z, a-z \\
%        & & $6$ & $2^{81}$ & $2^{77}$ & Весь алфавит \\
%        \hline
%    \end{tabular}
%    \caption{Сравнение верхней оценки числа паролей, построенных на слогах, со всем доступным множеством паролей в алфавите из $D$ символов.}
%    \label{tab:password-number}
%\end{table}
%
%Из таблицы~\ref{tab:password-number} видно, что при доступном объёме вычислений в $2^{60 \ldots 70}$ операций, пароли вплоть до 12 символов, построенные на словах, слогах, изменениях слов, вставках цифр, небольшого изменения регистров и другой простейшей обфускации, могут быть найдены перебором на кластере (или ПК) в настоящее время.


\subsubsection{Атака для подбора паролей и ключей шифрования}

В схемах аутентификации по паролю иногда используется хэширование и хранение хэша пароля на сервере. В таких случаях применима словарная атака или атака с применением заранее вычисленных таблиц для ускорения поиска.

Для нахождения пароля, прообраза хэш-функции, или для нахождения ключа блочного шифрования по атаке с выбранным шифртекстом (для одного и того же известного открытого текста и соответствующего шифртекста) может быть применён метод перебора с балансом между памятью и временем вычислений. Самый быстрый метод радужных таблиц\index{радужные таблицы} (\langen{rainbow tables}, 2003~г., \cite{Oechslin:2003}) заранее вычисляет следующие цепочки и хранит результат в памяти.

Для нахождения пароля, прообраза хэш-функции $H$, цепочка строится как
    \[ M_0 \xrightarrow{H(M_0)} h_0 \xrightarrow{R_0(h_0)} M_1 \ldots M_t \xrightarrow{H(M_t)} h_t \xrightarrow{R_t(h_t)} M_{t+1}, \]
где $R_i(h)$ -- функция редуцирования, преобразования хэша в пароль для следующего хэширования.

Для нахождения ключа блочного шифрования для одного и того же известного открытого текста $M$ таблица строится как
    \[ K_0 \xrightarrow{E_{K_0}(M)} c_0 \xrightarrow{R_0(c_0)} K_1 \ldots K_t \xrightarrow{E_{K_t}(M)} c_t \xrightarrow{R_t(c_t)} K_{t+1}, \]
где $R_i(c)$ -- функция редуцирования, преобразования шифртекста в новый ключ.

Функция редуцирования $R_i$ зависит от номера итерации, чтобы избежать дублирующихся подцепочек, которые возникают в случае коллизий между значениями в разных цепочках в разных итерациях, если $R$ постоянна. Радужная таблица размера $(m \times 2)$ состоит из строк $(M_{0,j}, M_{t+1,j})$ или $(K_{0,j}, K_{t+1,j})$, вычисленных для разных значений стартовых паролей $M_{0,j}$ или $K_{0,j}$ соответственно.

Опишем атаку на примере нахождения прообраза $\overline{M}$ хэша $\overline{h} = H(\overline{M})$. На первой итерации исходный хэш $\overline{h}$ редуцируется в сообщение $\overline{h} \xrightarrow{R_t(\overline{h})} \overline{M}_{t+1} $ и сравнивается со всеми значениями последнего столбца $M_{t+1,j}$ таблицы. Если нет совпадения, переходим ко второй итерации. Хэш $\overline{h}$ дважды редуцируется в сообщение $\overline{h} \xrightarrow{R_{t-1}(\overline{h})} \overline{M}_t \xrightarrow{H(\overline{M}_t)} \overline{h}_t \xrightarrow{R_t(\overline{h}_t)} \overline{M}_{t+1}$ и сравнивается со всеми значениями последнего столбца $M_{t+1,j}$ таблицы. Если не совпало, то переходим к третьей итерации и~т.\,д. Если для $r$-кратного редуцирования сообщение $\overline{M}_{t+1}$ содержится в таблице во втором столбце, то из совпавшей строки берётся $M_{0,j}$, и вся цепочка пробегается заново для поиска искомого сообщения $\overline{M}: ~ \overline{h} = H(\overline{M})$.

Найдём вероятность нахождения пароля в таблице. Пусть мощность множества всех паролей равна $N$. Изначально в столбце $M_{0,j}$ содержится $m_0 = m$ различных паролей. Предполагая наличие случайного отображения с пересечениями паролей $M_{0,j} \rightarrow M_{1,j}$, в $M_{1,j}$ будет $m_1$ различных паролей. Согласно задаче о размещении,
\[
    m_{i+1} = N \left( 1 - \left( 1 - \frac{1}{N} \right)^{m_i} \right) \approx N \left( 1 - e^{-\frac{m_i}{N}} \right).
\]
Вероятность нахождения пароля:
\[
    P = 1 - \prod \limits_{i=1}^t \left( 1 - \frac{m_i}{N} \right).
\]

Чем больше таблица из $m$ строк, тем больше шансов найти пароль или ключ, выполнив в наихудшем случае   $O \left( m \frac{t(t+1)}{2} \right)$ операций.

Примеры применения атаки на хэш-функциях $\textrm{MD5}$\index{хэш-функция!MD5}, $\textrm{LM} \sim \textrm{DES}_{\textrm{Password}} (\textrm{const})$ приведены в таблице~\ref{tab:rainbow-tables}.

\begin{table}[!ht]
    \centering
    \caption{Атаки на радужных таблицах на \emph{одном} ПК\label{tab:rainbow-tables}}
    \resizebox{\textwidth}{!}{ \begin{tabular}{|c|c|c|c|c|c|c|}
        \hline
        \multirow{2}{*}{\parbox{1.0cm}{\medskip\medskip \centering Длина, биты}} & \multicolumn{3}{|c|}{\parbox{4.3cm}{\medspace\centering Пароль или ключ}} &
            \multicolumn{3}{|c|}{\parbox{4.33cm}{\medspace\centering Радужная таблица}} \\
        \cline{2-7}
        & \parbox{1.0cm}{\centering Длина,\\ симв.} & \parbox{1.7cm}{\centering Множество} & \parbox{1.7cm}{\centering Мощность} &
            \parbox{1cm}{\centering Объём} & \parbox{2.23cm}{\medspace \centering Время вычисления таблиц} & \parbox{1.1cm}{\centering Время поиска} \\
        \hline \hline
        \multicolumn{7}{|c|}{Хэш LM} \\
        \hline
        \rule{0pt}{2.5ex}\multirow{3}{*}{$2 \times 56$} & \multirow{3}{*}{14} &
            A--Z & $2^{33}$ & 610 MB &  & 6 с \\
        & & A--Z, 0-9 & $2^{36}$ & 3 GB &  & 15 с \\
        & & все & $2^{43}$ & 64 GB & \parbox{2.23cm}{несколько лет} & 7 мин \\
        \hline \hline
        \multicolumn{7}{|c|}{Хэш MD5} \\
        \hline
        \rule{0pt}{2.5ex} 128 & 8 & A-Z, 0-9 & $2^{41}$ & 36 GiB & - & 4 мин \\
        \hline
    \end{tabular} }
\end{table}

\section{Аутентификация по паролю}

Из-за малой энтропии пользовательских паролей во всех системах регистрации и аутентификации пользователей применяется специальная политика безопасности. Типичные минимальные требования:
\begin{enumerate}
    \item Длина пароля от 8 символов. Использование разных регистров и небуквенных символов в паролях. Запрет паролей из словаря или часто используемых паролей. Запрет паролей в виде дат, номеров машин и других номеров.
    \item Ограниченное время действия пароля. Обязательная смена пароля по истечении срока действия.
    \item Блокирование возможности аутентификации после нескольких неудачных попыток. Ограниченное число актов аутентификации в единицу времени. Временная задержка перед выдачей результата аутентификации.
\end{enumerate}

Дополнительные меры предосторожности для пользователей:
\begin{enumerate}
    \item Не использовать одинаковые или похожие пароли для разных систем, таких как электронная почта, вход в ОС, электронная платёжная система, форумы, социальные сети. Пароль часто передаётся в открытом виде по сети. Пароль доступен администратору системы, возможны утечки конфиденциальной информации с серверов. Поэтому следует стараться выбирать случайные стойкие пароли.
    \item Не записывать пароли. Никому не сообщать пароль, даже администратору. Не передавать пароли по почте, телефону, Интернету и~т.\,д.
    \item Не использовать одну и ту же учётную запись для разных пользователей, даже в виде исключения.
    \item Всегда блокировать компьютер, когда пользователь отлучается от него, даже на короткое время.
\end{enumerate}

\input{os_passwords}

\input{http_auth}

\chapter{Программные уязвимости}

\input{security_models}

\input{os_access_controls}

\section{Виды программных уязвимостей}

\emph{Вирусом} называется самовоспроизводящаяся часть кода (подпрограмма)\index{вирус}, которая встраивается в носители (другие программы) для своего исполнения и распространения. Вирус не может исполняться и передаваться без своего носителя.

\emph{Червём} называется самовоспроизводящаяся отдельная (под)программа\index{червь}, которая может исполняться и распространяться самостоятельно, не используя программу-носитель.

Первой вехой в изучении компьютерных вирусов можно назвать 1949 год, когда Джон фон Нейман прочёл курс лекций в Университете Иллинойса под названием <<Теория самовоспроизводящихся машин>> (изданы в 1966~\cite{Neumann:1966}, переведены на русский язык издательством <<Мир>> в 1971 году~\cite{Neumann:1971}), в котором ввёл понятие самовоспроизводящихся механических машин. Первым сетевым вирусом считается вирус Creeper 1971 г., распространявшийся в сети ARPANET, предшественнице Интернета. Для его уничтожения был создан первый антивирус Reaper, который находил и уничтожал Creeper.

Первый червь для Интернета, червь Морриса, 1988 г., уже использовал \emph{смешанные} атаки\index{атака!смешанная} для заражения UNIX машин~\cite{EichinRochlis:1988, Spafford:1989}. Сначала программа получала доступ к удалённому запуску команд, эксплуатируя уязвимости в сервисах \texttt{sendmail}, \texttt{finger} (с использованием атаки на переполнение буфера) или \texttt{rsh}. Далее, с помощью механизма подбора паролей червь получал доступ к локальным аккаунтам пользователей:
\begin{itemize}
    \item получение доступа к учётным записям с очевидными паролями:
		\begin{itemize}
			\item без пароля вообще;
			\item имя аккаунта в качестве пароля;
			\item имя аккаунта в качестве пароля, повторённое дважды;
			\item использование <<ника>> (\langen{nickname});
			\item фамилия (\langen{last name, family name});
			\item фамилия, записанная задом наперёд;
		\end{itemize}
		\item перебор паролей на основе встроенного словаря из 432 слов;
		\item перебор паролей на основе системного словаря \texttt{/usr/dict/words}.
\end{itemize}

\emph{Программной уязвимостью}\index{программная уязвимость} называется свойство программы, позволяющее нарушить её работу. Программные уязвимости могут приводить к отказу в обслуживании (Denial of Service, DoS-атака)\index{атака!отказ в обслуживании}, утечке и изменению данных, появлению и распространению вирусов и червей.

Одной из распространённых атак для заражения персональных компьютеров является переполнение буфера в стеке. В интернет-сервисах наиболее распространённой программной уязвимостью в настоящее время является межсайтовый скриптинг (Cross-Site Scripting, XSS-атака)\index{атака!XSS}.

Наиболее распространённые программные уязвимости можно разделить на классы:
\begin{enumerate}
    \item Переполнение буфера -- копирование в буфер данных большего размера, чем длина выделенного буфера. Буфером может быть контейнер текстовой строки, массив, динамически выделяемая память и~т.\,д. Переполнение становится возможным вследствие либо отсутствия контроля над длиной копируемых данных, либо из-за ошибок в коде. Типичная ошибка -- разница в 1 байт между размерами буфера и данных при сравнении.
    \item Некорректная обработка (парсинг) данных, введённых пользователем, является причиной большинства программных уязвимостей в веб-приложениях. Под обработкой понимаются:
        \begin{enumerate}
            \item проверка на допустимые значения и тип (числовые поля не должны содержать строки и~т.\,д.);
            \item фильтрация и экранирование специальных символов, имеющих значения в скриптовых языках или применяющихся для перекодирования из одной текстовой кодировки в другую. Примеры символов: \texttt{\textbackslash}, \texttt{\%}, \texttt{<}, \texttt{>}, \texttt{"}, \texttt{'};
            \item фильтрация ключевых слов языков разметки и скриптов. Примеры: \texttt{script}, \texttt{JavaScript};
            \item перекодирование различными кодировками при парсинге. Распространённый способ обхода системы контроля парсинга данных состоит в однократном или множественном последовательном кодировании текстовых данных в шестнадцатеричные кодировки \texttt{\%NN} ASCII и UTF-8. Например, браузер или веб-приложения производят $n$-кратное перекодирование, в то время как система контроля делает $k$-кратное перекодирование, $0 \leq k < n$, и, следовательно, пропускает закодированные запрещённые символы и слова.
        \end{enumerate}
    \item Некорректное использование функций. Например, \texttt{printf(s)} может привести к уязвимости записи в память по указанному адресу. Если злоумышленник вместо обычной текстовой строки введёт в качестве \texttt{s "текст некоторой длины\%n"}, то функция \texttt{printf}, ожидающая первым аргументом строку формата \texttt{fmt}, обнаружив \texttt{\%n}, возьмёт значение из ячеек памяти, находящихся перед ячейками с указателем на текстовую строку (устройство стека описано далее), и запишет в память по адресу, равному считанному значению, количество выведенных символов на печать функцией \texttt{printf}.
\end{enumerate}


\input{stack_overflow}

\input{xss}

\input{sql-injections}

%\chapter{Послесловие}
%Это должно быть что-то в виде заключения, объяснения, почему именно эти темы выбраны, насколько актуален материал с теоретической и практической точки зрения.

\subimport*{appendix/}{index}

\printindex

\printbibliography[heading=bibintoc,title={Литература}]

\end{document}
