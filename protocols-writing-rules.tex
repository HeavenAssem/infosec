\section{Запись протоколов}
\selectlanguage{russian}
Для записи протоколов, связанных с реализацией функций защиты информации, не используют выражения вроде <<участник с ролью <<Отправитель>>>>, а заменяют их на краткие обозначения вроде <<отправитель>> или используют общепринятые экземплификанты\footnote{\emph{Экземплификант} или \emph{экземплификатив} -- конкретное понятие или имя собственное, используемое в качестве примера для обозначения неизвестного места или личности. (Википедия, свободная энциклопедия; 5 июля 2019)}: Алиса, Боб, Клара, Ева и~т.\,д. При этом используют следующие соглашения.

\begin{itemize}
	\item Алиса, Боб (от \langen{A, B}) -- отправитель и получатель.
	\item Карл, Клара, Чарли (от \langen{C}) -- равноправная третья сторона.
	\item Ева (от \langen{eavesdropper}) -- пассивный криптоаналитик.
	\item Меллори (от \langen{malicious}) -- активный криптоаналитик.
	\item Трент (от \langen{trust}) -- доверенная сторона.
\end{itemize}

Не существует общепринятого формата записи протоколов, они могут отличаться как по внешнему виду, так и по полноте описания. Например, вот наиболее полный формат записи протокола Диффи~---~Хеллмана\index{протокол!Диффи~---~Хеллмана}.

\begin{itemize}
	\item Предварительный этап.
	\begin{itemize}
		\item Все стороны выбрали общие $g$ и $p$.
	\end{itemize}
	\item Проход 1.
	\begin{itemize}
		\item Алиса генерирует случайное $a$.
		\item Алиса вычисляет $A = g^a \bmod p$.
		\item Алиса отправляет Бобу $A$.
	\end{itemize}
	\item Проход 2.
	\begin{itemize}
		\item Боб принимает от Алисы $A$.
		\item Боб генерирует случайное $b$.
		\item Боб вычисляет $B = g^b \bmod p$.
		\item Боб отправляет Алисе $B$.
		\item Боб вычисляет $s = A^b \bmod p$.
	\end{itemize}
	\item Проход 2.
	\begin{itemize}
		\item Алиса принимает от Боба $B$.
		\item Алиса вычисляет $s = B^a \bmod p$.
	\end{itemize}
	\item Результат протокола.
	\begin{itemize}
		\item Стороны вычислили общий сеансовый ключ $s$.
	\end{itemize}
\end{itemize}

Теперь сравните с краткой записью того же самого протокола.
\begin{enumerate}
	\item $A \to B : A = g^a \bmod p$
	\item $B \to A : B = g^b \bmod p$
\end{enumerate}

В краткой записи опускаются инициализация и предварительные требования, вычисления непередаваемых данных (в данном примере -- общего сеансового ключа $s$), а также любые проверки.

В данном пособии мы будем придерживаться промежуточного формата записи.

\begin{enumerate}
	\item[(1)] Алиса генерирует $a$.
	\item[] $Alice \to \left\{ A = g^a \bmod p \right\} \to Bob$.
	\item[(2)] Боб генерирует $b$.
	\item[] Боб вычисляет $s = A^b \bmod p$.
	\item[] $Bob \to \left\{ B = g^b \mod p \right\} \to Bob$.
	\item[(3)] Алиса вычисляет $s = B^a \bmod p$.
\end{enumerate}

Также условимся о правилах записи случая, когда активный криптоаналитик (Меллори) выдаёт себя за легального пользователя.

\[
\begin{array}{llllc}
(1) & A                & \to M \left(B\right) & : & A   = g^a     \bmod p, \\ 
(2) & M \left(A\right) & \to B                & : & A^* = g^{a^*} \bmod p, \\ 
(3) & B                & \to M \left(A\right) & : & B   = g^b     \bmod p, \\ 
(4) & M \left(B\right) & \to A                & : & B^* = g^{b^*} \bmod p. \\
\end{array}
\]

Либо, отводя отдельный столбец для каждого участника.
\[
\begin{array}{lllclllc}
	(1) & A  & \to   & M \left(B\right) & {}    & {} & : & A = g^a     \bmod p, \\ 
	(2) & {} & {}    & M \left(A\right) & \to   & B  & : & A^* = g^{a^*} \bmod p, \\ 
	(3) & {} & {}    & M \left(A\right) & \gets & B  & : & B   = g^b     \bmod p, \\ 
	(4) & A  & \gets & M \left(B\right) & {}    & {} & : & B^* = g^{b^*} \bmod p. \\
\end{array}
\]

Для сокращения описания и упрощения сравнения разных протоколов используют следующие соглашения об обозначениях передаваемых данных.

\begin{itemize}
	\item $M$ (от \langen{message}) -- сообщение в исходном виде, открытый текст вне зависимости от кодировки. То есть под $M$ может пониматься и исходный текст в виде текста или, например, звука, либо уже некоторое число или массив бит, однозначно соответствующие этому сообщению.
	\item $K$ (от \langen{key}) -- некоторый ключ. Без дополнительных уточнений обычно обозначает секретный сеансовый ключ.
	\item $K_A$ -- общий секретный ключ между Алисой и Трентом (для симметричных криптосистем).
	\item $K_A$ -- открытый ключ Алисы (для асимметричных криптосистем).
	\item $E_K \left( \dots \right)$ (от \langen{encrypt}) -- данные, зашифрованные на ключе $K$.
	\item $S_K \left( \dots \right)$ (от \langen{sign}) -- данные \emph{и} соответствующая цифровая подпись на открытом ключе $K$.
	\item $T_A$, $T_B$ (от \langen{timestamp}) -- метки времени от соответствующих участников.
	\item $R_A$, $R_B$ (от \langen{random}) -- случайные числа, выбранные соответствующими участниками.
\end{itemize}
