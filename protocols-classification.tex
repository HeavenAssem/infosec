\section{Классификация протоколов}\label{section-protocols-classification}
\selectlanguage{russian}

Общепризнанная классификация защитных протоколов отсутствует. Однако можно выделить набор \emph{объективных и однозначных} признаков, классифицирующих протоколы.

\begin{itemize}
    \item Классификация по числу участников протокола:
    \begin{itemize}
        \item двусторонний\index{протокол!двусторонний}, 
        \item трёхсторонний\index{протокол!трёхсторонний} и т.~п., 
        \item многосторонний\index{протокол!многосторонний}.
    \end{itemize}
    \item Классификация по числу передаваемых сообщений:
    \begin{itemize}
        \item интерактивный\index{протокол!интерактивный} (с наличием взаимного обмена сообщениями);
        \item неинтерактивный\index{протокол!неинтерактивный} (с однократной передачей сообщений), часто называется \emph{схемами}\index{схема}\footnote{Определение не совсем полное. Любая схема предполагает как минимум два этапа. На первом предварительном этапе доверенный центр распределяет некоторую информацию между одноранговыми участниками. На втором этапе (конкретные сеансы протокола) участники обмениваются этой информацией, получая исходный секрет или новый секретный сеансовый ключ. Причём обмен информацией может идти более чем между двумя участниками. Однако после взаимного обмена информацией дополнительных проходов для выполнения целей схемы не требуется.}.
    \end{itemize}
    \item Классификация по числу проходов (раундов):
    \begin{itemize}
        \item двупроходной (двураундовый),
        \item трёхпроходной (трёхраундовый) и т.~д.,
        \item многопроходной (многораундовый) или циклический.
    \end{itemize}
    \item Классификация по используемым криптографическим системам:
    \begin{itemize}
        \item на основе только симметричных\index{криптосистема!симметричная} криптосистем;
        \item на основе в том числе асимметричных\index{криптосистема!асимметричная} криптосистем.
    \end{itemize}
    \item Классификация по защищённым свойствам протокола:
    \begin{itemize}
        \item[(G1)] обеспечивает или нет аутентификацию первой, второй стороны протокола и т.~д.;
        \item[(G2)] обеспечивает или нет аутентификацию сообщений;
        \item[(G3)] обеспечивает или нет защиту от повторов;
        \item[{}] и т.~п.
    \end{itemize}
    \item Классификация по типам участников:
    \begin{itemize}
        \item одноранговый, когда все участники могут выполнять любые роли в рамках протокола;
        \item с доверенным посредником, когда в протоколе всегда участвует третья доверенная сторона;
        \item с доверенным арбитром, когда в протоколе может участвовать третья доверенная сторона, если остальные участники не пришли к согласию.
    \end{itemize}
\end{itemize}

Можно также ввести менее объективную и однозначную классификацию, основываясь на субъективной оценке протоколов.
\begin{itemize}
    \item Классификация по целевому назначению протокола:
    \begin{itemize}
        \item \dots обеспечения целостности, 
        \item \dots цифровой подписи, 
        \item \dots идентификации, 
        \item \dots конфиденциальности, 
        \item \dots распределения ключей, 
        \item \dots и~т.~п.
    \end{itemize}
    \item Классификация по <<полноте>> выполняемых функций:
    \begin{itemize}
        \item примитивные, используются как базовый компонент при построении прикладных протоколов;
        \item промежуточные;
        \item прикладные, предназначены для решения практических задач.
    \end{itemize}
\end{itemize}

Классификацию по целевому предназначению можно также переформулировать в виде классификации по защищённым свойствам протокола, для обеспечения которых он разрабатывался. При этом нужно будет выделить <<основные свойства>> (например, G10 -- формирование новых ключей), а большую часть остальных отнести к дополнительным (например, G7 -- аутентификация ключа, и G8 -- подтверждение владения ключом). Определение того, какие именно из свойств <<основные>>, а какие <<дополнительные>>, будет создавать неоднозначность классификации по целевому назначению протокола. Если же все свойства протокола назвать <<основными>>, то такая классификация станет слишком детальной.

Классификация по <<полноте>> выполняемых функций проблематична из-за того, что ни один протокол нельзя назвать в полной мере <<прикладным>>. Любой протокол сам по себе это лишь часть некоторой информационной (или организационной) системы, которая как раз и выполняет требуемую пользователями функцию. Однако можно говорить о том, что отдельные протоколы (например, TLS\index{протокол!TLS}) являются протоколами более высокого уровня, чем протоколы, например, Диффи~---~Хеллмана\index{протокол!Диффи~---~Хеллмана}, так как последний часто выступает составной частью того же протокола TLS.
