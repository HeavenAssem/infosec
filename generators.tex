\chapter{Генераторы псевдослучайных чисел}\label{chapter-generators}
\selectlanguage{russian}

Для работы многих криптографических примитивов необходимо уметь получать случайные числа:
\begin{itemize}
	\item вектор инициализации для отдельных режимов сцепления блоков должен быть случайным числом (см. раздел~\ref{section-block-chaining});
	\item для генерации пар открытых и закрытых ключей необходимы случайные числа (см. главу~\ref{chapter-public-key});
	\item стойкость многих криптографических протоколов ключей (см. главу~\ref{chapter-protocols}) основывается в том числе на выработке случайных чисел (\langen{nonce}), которые не может предугадать злоумышленник.
\end{itemize}

Генератором случайных чисел (\langen{random number generator})\index{генератор!случайных чисел} мы будем называть процесс\footnote{Есть и строгое математическое определение генератора в общем смысле. Генератором называется функция $g: \left\{0, 1\right\}^{n} \to \left\{0, 1\right\}^{q\left(n\right)}$, вычислимая за полиномиальное время. Однако мы пока не будем использовать это определение, чтобы показать разницу между истинно случайными числами и псевдослучайными.}, результатом работы которого является случайная последовательность чисел, а именно такая, что зная произвольное число предыдущих чисел последовательности (и способ их получения), даже теоретически нельзя предсказать следующее с вероятностью больше заданной. К таким случайным процессам можно отнести:

\begin{itemize}
	\item результат работы счётчика элементарных частиц, работа с которым включена в лабораторный практикум по общей физике для студентов первого курса МФТИ;
	\item время между нажатиями клавиш на клавиатуре персонального компьютера или расстояние, которое проходит <<мышь>> во время движения;
	\item время между двумя пакетами, полученными сетевой картой;
	\item тепловой шум, измеряемый звуковой картой на входе аналогового микрофона, даже при отсутствии самого микрофона.
\end{itemize}

Хотя для всех этих процессов можно предсказать приблизительное значение (чётное или нечётное), его последний бит будет оставаться достаточно случайным для практических целей. С учётом данной поправки их можно называть надёжными или качественными генераторами случайных чисел.

Однако к генератору случайных чисел предъявляются и другие требования. Кроме уже указанного критерия \emph{качественности} или \emph{надёжности}, генератор должен быть \emph{быстрым} и \emph{дешёвым}. Быстрым -- чтобы получить большой объём случайной информации за заданный период времени. И дешёвым -- чтобы его можно было бы использовать на практике. Количество случайной информации от перечисленных выше генераторов составляет не более десятков килобайт в секунду (для теплового шума) и значительно меньше, если мы будем требовать ещё и равномерность распределения полученных случайных чисел.

С целью получения большего объёма случайной информации используют специальные алгоритмы, которые называют генераторами псевдослучайных чисел (ГПСЧ). ГПСЧ -- это детерминированный алгоритм, выходом которого является последовательность чисел, обладающая свойством случайности. Работу ГПСЧ можно описать следующей моделью. На подготовительном этапе оперативная память, используемая алгоритмом, заполняется начальным значением (\langen{seed}). Далее на каждой итерации своей работы ГПСЧ выдаёт на выход число, которое является функцией от состояния оперативной памяти алгоритма, и меняет содержимое своей памяти по определённым правилам. Содержимое оперативной памяти называется \emph{внутренним состоянием} генератора.

Как и у любого алгоритма, у ГПСЧ есть определённый размер используемой оперативной памяти\footnote{Только алгоритмы с фиксированным размером используемой оперативной памяти и можно называть \emph{генераторами} в строгом математическом смысле этого слова, как следует из определения.}. Исходя из практических требований, предполагается, что размер оперативной памяти для ГПСЧ сильно ограничен. Так как память алгоритма ограничена, то ограничено и число различных внутренних состояний алгоритма. В силу того, что выдаваемые ГПСЧ числа являются функцией от внутреннего состояния, то любой ГПСЧ, работающий с ограниченным размером оперативной памяти и не принимающий извне дополнительной информации, будет иметь \emph{период}. Для генератора с памятью в $n$ бит максимальный период, очевидно, равен $2^n$.

Качество детерминированного алгоритма, то есть то, насколько полученная последовательность обладает свойством случайной, можно оценить с помощью тестов, таких как набор тестов NIST (\langen{National Institute of Standards and Technology}, США,~\cite{NIST:2001}). Данный набор содержит большое число различных проверок, включая частотные тесты бит и блоков, тесты максимальных последовательностей в блоке, тесты матриц и так далее.

\input{linear-congruential-generator}

\input{lfsr}

\input{crypto-random}

\input{bbs_generator}

\section{КСГПСЧ на основе РСЛОС}

Как уже упоминалось ранее, использование РСЛОС в качестве ГПСЧ не является криптографически стойким. Однако можно использовать комбинацию из нескольких регистров сдвига, чтобы в результате получить быстрый, простой (дешёвый) и надёжный (криптографически стойкий) генератор псевдослучайных чисел.

\input{generators_with_multiple_shift_registers}

\input{generators_with_nonlinear_transformations}

\input{majority_generators}
